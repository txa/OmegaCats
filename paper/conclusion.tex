\section{Conclusions and Further Work}
\label{sec:conclusions}

We have presented a novel approach to defining weak $\omega$-groupoids
which is based on ideas from Type Theory. The central idea is to
define the syntax of weak $\omega$-groupoids and then define a weak
$\omega$-groupoid as a globular set with an interpretation of the
syntax, which is where Type Theory has its greatest strength. Indeed,
we have formalized most of the material presented here in Agda
\cite{agda}.  We believe that our approach to formalization of
coherence is natural, in a way naive, since it is a natural
generalisation of the corresponding first order laws.

The current formalisation is still
quite complicated and we hope to find ways to simplify it. One interesting
idea may be to use the syntactical approach to define opetopes based on
dependent polynomial functors (i.e. indexed containers) \cite{opetopes},
which has a very type-theoretic flavour.  

We would like to use our framework to provide a formalisation of a
variation of the results by in
\cite{lumsdaine10:weak-o-categories,berg08:types-are} by showing that
$\Idw$ is a weak $\omega$-groupoid. This is slightly different form
their results because we are working insinde Type Theory instead of on
a metalevel.

The main challenge ahead is to formalize the notion of an
$\omega$-groupoid model of Type Theory. Once this has been done we
will be able to eliminate the univalence axiom and provide a
computational interpretation of this principle.


%%% Local Variables: 
%%% mode: latex
%%% TeX-master: "weakomega2"
%%% End: 

