\section{Conclusions and Further Work}
\label{sec:conclusions}

\subsection{Summary}
We have presented a novel approach to defining weak $\omega$-groupoids
which is based on ideas from Type Theory. The central idea is to
define the syntax of weak $\omega$-groupoids and then define a weak
$\omega$-groupoid as a globular set with an interpretation of the
syntax, which is where Type Theory has its greatest strength. Indeed,
we have formalized most of the material presented here in Agda
\cite{agda}.  We believe that our approach to formalization of
coherence is natural, in a way naive, since it is a natural
generalisation of the corresponding first order laws.

\subsection{Related work}
There exists an abundance of categorical definitions of weak
$\omega$-categories and groupoids.  Although a direct comparison is a
slippery road, we would like to give a rough comparison of the key
similarities and differences between our and other definitions. Most
importantly, our definition is fundamentally different in that it is
formulated in Type Theory rather than Category Theory. This means that
we couldn't just formalise any of the approaches
\cite{penon:1999,batanin98:monoidal-globular,leinster:2000} because at
their heart is to the notion of a strict $\omega$-category which
drives the definition of coherence cells. However, the notion of a
\emph{strict} $\omega$-category is not available in Type Theory
without quotient types. This forces some of the choices we have
made. Nevertheless, at an intuitive level there are similarities of our approach to the
categorical approaches, some of which we mention below.
\begin{itemize}
\item Penon's definition \cite{penon:1999}, similarly to us, uses
  binary rather than unbiased composition, and explicit
  identities. 
\item As we can't form quotients and compare for strict equality in
  the underlying strict monoidal set, we, similarly to Street
  \cite{street87:simplexes} generate coherence cells
  inductively. 
\item Batanin's \emph{spans} are essentially our telescopes. However
  his composition is unbiased and therefore he trees in spans, which
  are definable but of no importance to us. 
\end{itemize}
%
As far as we know, there is no single definition that would
integrate all these points into a single definition. 



\subsection{Further work}
The current formalisation is still
quite complicated and we hope to find ways to simplify it. One interesting
idea may be to use the syntactical approach to define opetopes based on
dependent polynomial functors (i.e. indexed containers) \cite{opetopes},
which has a very type-theoretic flavour.  

We would like to use our framework to provide a formalisation of a
variation of the results by in
\cite{lumsdaine10:weak-o-categories,berg08:types-are} by showing that
$\Idw$ is a weak $\omega$-groupoid. This is slightly different form
their results because we are working inside Type Theory rather than on
a meta-level.

The main challenge ahead is to formalize the notion of an
$\omega$-groupoid model of Type Theory. Once this has been done we
will be able to eliminate the univalence axiom and provide a
computational interpretation of this principle.


%%% Local Variables: 
%%% mode: latex
%%% TeX-master: "weakomega2"
%%% End: 

