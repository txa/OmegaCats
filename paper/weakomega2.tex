\documentclass[conference, a4paper]{IEEEtran}
%\documentclass[a4paper]{article}

\usepackage{latexsym, amsmath, amssymb}
\usepackage{amsthm}
\RequirePackage{mathrsfs}
\RequirePackage{stmaryrd} %boxes
\usepackage{diagxy}
\usepackage{bussproofs}
\input{macros.tex}


\begin{document}
\title{A Syntactical Approach to Weak $\omega$-Groupoids}

% author names and affiliations
% use a multiple column layout for up to three different
% affiliations
\author{\IEEEauthorblockN{Thorsten Altenkirch}
\IEEEauthorblockA{Functional Programming Laboratory\\
School of Computer Science\\
University of Nottingham, UK}
\and
\IEEEauthorblockN{Ondrej Rypacek}
\IEEEauthorblockA{Department of Informatics\\
King's College London\\
London, UK
Email:ondrej.rypacek@kcl.ac.uk}
}

\maketitle


\begin{abstract}
%\boldmath
  When moving to a Type Theory without proof irrelevance the notion of
  a setoid has to be generalized to the notion of a weak
  $\omega$-groupoid. As a first step in this direction we study the
  formalisation of weak $\omega$-groupoids in Type Theory. This is
  motivated by Voevodsky's proposal of univalent type theory which is
  incompatible with proof-irrelevance and the results by Lumsdaine and
  Garner/van de Berg showing that the standard eliminator for equality
  gives rise to a weak $\omega$-groupoid.
\end{abstract}
\IEEEpeerreviewmaketitle

\section{Introduction}

The main motivation for the present work is the development of
Univalent Type Theory by Voevodsky and others \cite{voevodsky}. In a
nutshell, Univalent Type Theory is a variant of Martin-L\"of's Type
Theory where we give up the principle of uniqueness of identity proofs
(UIP) to make it possible to treat equivalence of structures
(e.g. isomorphism of sets) as equality. While Voevodsky's motivation
comes from Homotopy Theory, Univalent Type Theory has an intrinsic
type theoretic motivation in enabling us to treat abstract structures
as first class citizen making it possible to combine high level
reasoning and concrete applicationswithout unnecessary clutter. 

The central principle of Univalent Type Theory is the Univalence Axiom
which states that equality of types is weakly equivalent to weak
equivalence. Here weak equivalence is a notion motivated by homotopy
theoretic models of type theory which can be alternatively understood
as a refinement of the notion of isomorphism in the absence of
UIP. The Univalence axiom can be viewed as a strong extensionality
priniciple and indeed it implies functional extensionality. As with
functional extensionality, univalence doesn't easily fit within the
computational understanding of Type Theory, since it does not fit into
the common pattern of introduction and elimination rules. The first
author has suggested a solution of this problem for functional
extensionality \cite{alti:lics99}: we can justify extensionality by a
translation based on the setoid model. This approach was later refined
\cite{plpv08} to \emph{Observational Type Theory} which is the base
for the development of Epigram 2 \cite{epigram2}.

However, Observational Type Theory relies essentially on UIP and hence
is incompatible with Univalent Type Theory. To address this we need to
replace setoids with a structure able to model non-unique identity
proofs. A first step in this direction is the groupoid model
\cite{HS:groupoid} but this forces UIP on the next level, i.e. for
equality between equality proofs. 
\footnote{\cite{HarperLicata} have shown that the appropriate
  restriction of Univalence can be eliminated in this setting.}
To be able to model Type Theory
without UIP at any level we need to move to
$\omega$-groupoids. Moreover, the equalities we need to assume are in
general non-strict (i.e. they are not definitional equalities in the
snese of Type Theory) and hence we need to look at weak
$\omega$-groupoids. Indeed, as \cite{garner} and \cite{lumsdaine} have
shown: Type Theory with Identity types gives rise to a weak
$\omega$-groupoid. 

Our goal is to eliminate unrestricted univalence by formalizing a weak
$\omega$-groupoid model of Type Theory in Type Theory.  As a first
step we need to implement the notion of a weak $\omega$-groupoid in
Type Theory and this is what we do in the present paper.  An obvious
possibility would have been to implement a categorical notion of weak
$\omega$ groupoids (eg. based on globular operads) in Type
Theory. However, this forces us to implement many categorical notions
first generating an avoidable overhead. It also seems that a structure
with a more typetheoretic flavour is more accessible from a naive
point of view. Hence, we are looking for a more direct
type-theoretic formulation of weak $\omega$-groupoids. In the present
paper we attempt this by defining a weak $\omega$-groupoid to be a
globular set with additional structure where this structure is given
by interpreting a syntactic theory in the globular set.



% \begin{quote}
%   \begin{itemize}
%   \item 
%     Why, potential applications: eliminating univalence axiom
%   \item cite: Harber\& Licata, Lumsdene
%   \item ref to opetopes paper (related work) Batanin
%   \end{itemize}
% \end{quote}

%%% Local Variables: 
%%% mode: latex
%%% TeX-master: "weakomega2"
%%% End: 


\section{Globular Sets \textit{(txa)}}

\begin{quote}
  \begin{itemize}\item coinductive defn
  \item presheaf?
  \item define Id$\omega$
  \end{itemize}
 \end{quote}
 


\section{Syntax}\label{sec:syntax}

% \begin{quote}
%   \begin{itemize}
%   \item generic syntax
%   \item Con,Cat,Obj,Var,Tel
%   \item Interpretation wrt globular sets
%   \item Syntax => globular set
%   \end{itemize}
% \end{quote}

Our goal is to specify the conditions under which a globular set is a
weak $\omega$-groupoid. This means we need to require the existence of
certain objects in various object sets within the structure. A natural
way would be to generalize the definition of a setoid and add these
components to the structure. However,  it is not clear how to capture
the coherence condition which basically says that any two morphisms
which just represent identities in the strict case should be
equal. Instead we will follow a different approach which can be
compared to the definition of environment models for the
$\lambda$-calculus: we shall define a syntax for weak
$\omega$-groupoids and then define a weak $\omega$-groupoid as a
globular set in which this syntax can be interpreted.

\subsection{The syntactical framework}
\label{sec:syntactical-framework}

We start by presenting a syntactical framework which is a syntax for
globular sets. This syntax could be used to identify any globular set
with structure (e.g. weak or strict $\omega$-categories), the specific
aspects of a weak $\omega$-groupoid will be introduced later by adding
additional syntax for objects and other auxiliary syntactic
components. 

Our framework consists of the following main components which are defined
by mutual induction\footnote{This is an instance of an inductive-inductive
definition in Type Theory, see \cite{alti:catind2}.}:
\begin{description}
\item[Contexts] 
\[\mathsf{Con} : \mathsf{Set} 
% \qquad
% \frac{\Gamma~:~\Con}{\GlobSet~\Gamma ~:~\Set}
% \qquad
% \AxiomC{$G : \GlobSet~\Gamma$}
% \UnaryInfC{$\mathsf{Var}~G : \mathsf{Set}$}
% \DisplayProof
\]
Contexts serve to formalize the existence of hypothetical objects
which are specified by the globular set in which they live.  E.g. to
formalize ordinary composition we have to assume that objects $a,b,c$
and 1-cells $f : a \to b$ and $g : b \to c$ exist to be able to form
$g \circ f : a \to c$.
\item[Categories] 
\[
\frac{\Gamma~:~\Con}{\GlobSet~\Gamma ~:~\Set}
\qquad
\AxiomC{$\Gamma : \mathsf{Con}$}
\UnaryInfC{$\mathsf{Cat}~\Gamma : \mathsf{Set}$}
\DisplayProof
\]
In order to define the valid compositions of cells one needs to know
their boundaries, i.e. iterated domains and codomains in the globular
case. Category expressions record this data. 
We define two kinds of categories: $\GlobSet$s are categories which
contain only variables, while $\Cat$s contain all cells freely
generated from variables. 
The set of expressions for both kinds of categories depends on a context, e.g. we
need at least to assume that there is one object in the top-level
category to be able to form any other categories.
\item[Variables \& Objects] 
\[
\frac{G : \GlobSet~\Gamma}
{\mathsf{Var}~G : \mathsf{Set}}
\qquad
\AxiomC{$C : \mathsf{Cat}~\Gamma$}
\UnaryInfC{$\mathsf{Obj}~C:\mathsf{Set}$}
\DisplayProof
\]
$\GlobSet$s contain only variables, which are projections out of the
context $\Gamma$. On the other hand, 
given a category we define all expressions which identify objects lying
within the category.  As indicated above this is the main focus of the
forthcoming sections. 



\end{description}
We now specify the constructors for the various sets (apart from
objects). We use unnamed variables ala deBruijn, hence contexts are
basically sequences of categories. However, note that this is a
dependent context since the well-formedness of a category expression
may depend on the previous context. At the same time we build globular
sets from nameless variables in contexts. 
\[
\AxiomC{\mathstrut}
\UnaryInfC{$\varepsilon : \mathsf{Con}$}
\DisplayProof
\qquad
\AxiomC{$G : \GlobSet~\Gamma$}
\UnaryInfC{$(\Gamma , G) : \mathsf{Con}$}
\DisplayProof
\qquad
\frac{}{\bullet : \GlobSet~\Gamma}\qquad\frac{G :
  \GlobSet~\Gamma\quad a , b : \mathsf{Var} ~G}{G [ a , b ] : \GlobSet~\Gamma}
\]
\[
\AxiomC{$\phantom{I}$}
\dblline
\UnaryInfC{$\mathsf{vz}:\mathsf{Var}~(\mathsf{wk}~G)$}
\DisplayProof
\qquad
\AxiomC{$v : \mathsf{Var}~G$}
\dblline
\UnaryInfC{$\mathsf{vs}~v : \mathsf{Var}~(\mathsf{wk}~G~G')$}
\DisplayProof
\]
where $\mathsf{wk}$ is weakening defined for categories by
induction on the structure in the obvious way: 
\[
\AxiomC{$G,\,G' : \GlobSet~\Gamma$}
\UnaryInfC{$\mathsf{wk}~G~G' : \GlobSet~(\Gamma,G')$}
\DisplayProof
\qquad
\begin{array}{l}
\mathsf{wk}~\bullet~G'\,=\,\bullet\\
\mathsf{wk}~(\homcat{G}{a}{b})~G'\,=\,(\mathsf{wk}~G~G')[\mathsf{vs}~a,\mathsf{vs}~b]
\end{array}
\]
There are two ways to form category expressions: there is the top
level category denoted by $\bullet$ and given any two objects 
$a,b$ in a category $C$ we can form the hom category $C[a,b]$.
\[
\AxiomC{$\phantom{\Gamma}$}
\UnaryInfC{$\bullet : \mathsf{Cat}~\Gamma$}
\DisplayProof
\qquad
\AxiomC{$C : \mathsf{Cat}~\Gamma\quad a,\,b : \mathsf{Obj}~C$}
\UnaryInfC{$C[ \,a\,,\,b\,] : \mathsf{Cat}~\Gamma$}
\DisplayProof
\]

% from weakening for objects:
% \[
% \AxiomC{$x : \mathsf{Obj}~C$}
% \AxiomC{$D:\Cat~\Gamma$}
% \BinaryInfC{$\mathsf{wk}~x~D:\mathsf{Obj}~(\mathsf{wk}~C~D)$}
% \DisplayProof
% \]
% which is a constructor of $\mathsf{Obj}$. 

\noindent Variables become objects by the following constructor of
$\Obj$, which mutually extends to $\GlobSet$s:
\[
\AxiomC{$v : \mathsf{Var}~G$}
\dblline
\UnaryInfC{$\mathsf{var}~v : \mathsf{Obj}~(\mathsf{var}~G)$}
\DisplayProof
\qquad
\frac{G : \GlobSet~\Gamma}
{\mathsf{var}~G : \Cat~\Gamma}
\quad\text{where}\quad
\begin{array}{lcl}
\mathsf{var}~\bullet&=& \bullet\\
\mathsf{var}~\homcat{G}{a}{b}&=&\homcat{(\mathsf{var}~G)}{\mathsf{var}~a}{\mathsf{var}~b}
\end{array}
\]


We use the usual arrow notation for categories and objects. For
instance, $\bullet[a,b]$, $\bullet[a,b][f,g]$ and $\alpha :
\Obj~(\bullet[a,b][f,g])$ are pictured respectively as follows:
\[\bfig
\morphism/{}/<300,0>[a`b;]
\efig
\quad\qquad 
\bfig
\morphism/{@{>}@/^1em/}/[a`b;f]
\morphism|b|/{@{>}@/_1em/}/[a`b;g]
\efig
\qquad 
\bfig
\morphism/{@{>}@/^1em/}/[a`b;f]
\morphism|b|/{@{>}@/_1em/}/[a`b;g]
%\morphism(250,80)|r|<0,-140>[`;\alpha]
\place(250,0)[\Downarrow]
\place(310,-20)[{^\alpha}]
\efig
\]
%
We also write, as usual, $x : a_n\longrightarrow b_n : \cdots : a_0
\longrightarrow b_0$ for an $x :
\Obj~(\bullet[a_0,b_0]\cdots[a_n,b_n])$.  Note that it is essential to
first introduce $\GlobSet$s and then $\Cat$s with an inclusion
\[
\mathsf{var}:\Sigma(\GlobSet~\Gamma)~\mathsf{Var} \to \Sigma(\Cat~\Gamma)~\Obj\]
In this way we make
sure that variables alone form a globular set, i.e. that the domain
and codomain of a variable is a variable. In particular, that it is not
possible to introduce a variable between syntactically constructed
coherence cells. In this way we can talk about the $\omega$-category
freely generated by a globular set.

% This is the basic setup for the syntax of weak omega categories,
% obviously more constructors for $\mathsf{Obj}$ are needed which we
% will discussed in the rest of the text. 

\subsection{Interpretation}
\label{sec:interpretation}
Given a globular set we define what we mean by an interpretation of
the syntax. Once we have specified all the constructors for objects a
weak $\omega$-groupoid is given by such an interpretation. The
interpretation of the structural components given in the present
section is fixed. Again this is reminiscent of environment models.

An \emph{interpretation} in a globular set $G:\Glob$ is given by the
following data:
\begin{enumerate}
\item An assignment of sets to contexts:
\[
% %<<<<<<< HEAD
% \AxiomC{$o : \Obj~C\qquad x : \intpr{\Gamma}$}
% \UnaryInfC{$\intpr{o}~x : \mathsf{obj}~(\intpr{C}~x)$}
% %=======
\AxiomC{$\Gamma : \Con$}
\UnaryInfC{$\intpr{\Gamma} : \Set$}
%>>>>>>> 341cc994e3d493b5de0c47333fc330bf43f3a890
\DisplayProof
\]
\item An assignment of globular sets to $\GlobSet$ and $\Cat$ expressions:
\[
\AxiomC{$G : \GlobSet ~ \Gamma$}
\AxiomC{$\gamma : \intpr{\Gamma}$}
\BinaryInfC{$\intpr{G}~\gamma : \Glob$}
\DisplayProof
\qquad
\AxiomC{$C : \Cat ~ \Gamma$}
\AxiomC{$\gamma : \intpr{\Gamma}$}
\BinaryInfC{$\intpr{C}~\gamma : \Glob$}
\DisplayProof
\]
\item An assignment of elements of object sets to object
  expressions and variables
\[
\AxiomC{$G : \GlobSet ~ \Gamma$}
\AxiomC{$x : \mathsf{Var}~G$}
\AxiomC{$\gamma : \intpr{\Gamma}$}
\TrinaryInfC{$\intpr{x}~\gamma : \obj_{\intpr{G}~\gamma}$}
\DisplayProof
\qquad
\AxiomC{$C : \Cat ~ \Gamma$}
\AxiomC{$A : \mathsf{Obj}~C$}
\AxiomC{$\gamma : \intpr{\Gamma}$}
\TrinaryInfC{$\intpr{A}~\gamma : \obj_{\intpr{C}~\gamma}$}
\DisplayProof
\]
\end{enumerate}
subject to the following conditions:
\[\begin{array}{lclclcl}
\intpr{\varepsilon}  & = & 1 &\quad&\intpr{\mathsf{var}~x}~\gamma  & =&  \intpr{x}~\gamma \\
\intpr{\Gamma , G} & =  &\Sigma \gamma : \intpr{\Gamma},\intpr{G} ~
\gamma&&\intpr{\mathsf{vz}}~(\gamma,a)  & = & a \\   % &&\intpr{\wk~a}~(\gamma,b)  & = &\intpr{a}~\gamma\\
\intpr{\bullet}~\gamma & = & G&&\intpr{\mathsf{vs}\,x}~(\gamma,a)  & = &\intpr{x}~\gamma\\
\intpr{\homcat{C}{a}{b}}~\gamma & = & \mathsf{hom}_{\intpr{C}
  \gamma}~(\intpr{a} ~ \gamma)~(\intpr{b}~\gamma)% && \intpr{\homcat{G}{a}{b}}~\gamma &= & \mathsf{hom}_{\intpr{G}  \gamma}~(\intpr{a} ~ \gamma)~(\intpr{b}~\gamma)
\end{array}
\enspace,\]
where the last case applies both to $\GlobSet$s and $\Cat$s.

%%% Local Variables: 
%%% mode: latex
%%% TeX-master: "weakomega-csl"
%%% End: 


\section{Structure}
\label{sec:structure}



A category, strict or weak, is a globular set with additional
structure. The difference between the strict and the weak case is weather
we adorn the structure with (equational) constraints or whether one instead
of axioms introduces more structure, which witnesses rather than postulates
the constraints; so-called \emph{coherence cells}. In this section we
introduce the syntax for the structure of composition and units
giving rise to syntax for what one could call a \emph{pre-monoidal
  globular category}, where composites and units are expressible but
unconstrained by coherence cells.

\subsection{Composition}\label{sec:composition}
%
\newcommand{\cC}{\mathcal{C}}
%
In the ordinary case, a category, $\mathcal{C}$, defines a partial
operation of composition on its arrows. Explicitly, for $a$, $b$, $c$
objects of $\mathcal{C}$, $f$ in $\homcat{\mathcal{C}}{a}{b}$, $g$  in
$\homcat{\mathcal{C}}{b}{c}$, there is a $gf$ in $\homcat{\mathcal{C}}{a}{c}$.
%
In the higher-dimensional case, $\cC(a,b)$ and $\cC(b,c)$ are not mere
sets but $\omega$-categories and composition extends from sets the
whole hom-categories. Informally: for $a$, $b$, $c$ as before, $f$,
$g$, $n$-cells of homcategories $\cC(a,b)$ and $\cC(b,c)$,
respectively, one requires an $n$-cell $g\circ f$ of $\cC(a,c)$. The
fact that both $f$ and $g$ are of the same relative depth with respect
to $\cC$ is important, as well the fact the homcategories of $f$ and $g$
meet at a common object, $b$, of $\cC$. Following are some
examples of valid compositions for increasing $n$:
\begin{align}
\label{eq:comp1}
\bfig
\morphism[a`b;f]
\morphism(500,0)[b`c;g]
\efig
&\quad\mapsto\quad
\bfig
\morphism[a`c;gf]
\efig
\\
\label{eq:comp2}
\bfig
\morphism/{@{>}@/^1em/}/[a`b;f]
\morphism|b|/{@{>}@/_1em/}/[a`b;f']
\morphism(250,50)<0,-100>[`;\alpha]
\morphism(500,0)/{@{>}@/^1em/}/[b`c;g]
\morphism(500,0)|b|/{@{>}@/_1em/}/[b`c;g']
\morphism(750,50)<0,-100>[`;\beta]
\efig
&\quad\mapsto\quad
\bfig
\morphism/{@{>}@/^1em/}/[a`c;gf]
\morphism|b|/{@{>}@/_1em/}/[a`c;g'f']
\morphism(250,50)<0,-100>[`;\beta\alpha]
\efig
% \label{eq:comp3}
% \bfig\scalefactor{1.2}
% \morphism/{@{>}@/^1em/}/[a`b;f]
% \morphism|b|/{@{>}@/_1em/}/[a`b;f']
% \morphism(180,50)/{@{>}@/_.1em/}/<0,-100>[`;\alpha]
% \morphism(320,50)|r|/{@{>}@/^.1em/}/<0,-100>[`;\alpha']
% \morphism(500,0)/{@{>}@/^1em/}/[b`c;g]
% \morphism(500,0)|b|/{@{>}@/_1em/}/[b`c;g']
% \morphism(680,50)/{@{>}@/_.1em/}/<0,-100>[`;\beta]
% \morphism(820,50)|r|/{@{>}@/^.1em/}/<0,-100>[`;\beta']
% \morphism(200,0)<100,0>[`;\gamma]
% \morphism(700,0)<100,0>[`;\delta]
% \efig
% &\quad\mapsto\quad
% \bfig
% \morphism/{@{>}@/^1em/}/[a`c;gf]
% \morphism|b|/{@{>}@/_1em/}/[a`c;g'f']
% \morphism(180,50)/{@{>}@/_.1em/}/<0,-100>[`;\beta\alpha]
% \morphism(320,50)|r|/{@{>}@/^.1em/}/<0,-100>[`;\beta'\hspace{-3pt}\alpha']
% \morphism(200,0)<100,0>[`;\delta\gamma]
% \efig\\
\end{align}
\begin{equation}
\label{eq:comp4}
\bfig\scalefactor{1.2}
\morphism/{@{>}@/^1.7em/}/[a`b;f]
\morphism|m|[a`b;f']
\morphism(180,110)/{@{>}@/_.1em/}/<0,-100>[`;\alpha]
\morphism(320,110)|r|/{@{>}@/^.1em/}/<0,-100>[`;\alpha']
\morphism(200,55)<100,0>[`;\gamma]
%
\morphism|b|/{@{>}@/_1.7em/}/[a`b;f'']
\morphism(180,-10)/{@{>}@/_.1em/}/<0,-100>[`;\beta]
\morphism(320,-10)|r|/{@{>}@/^.1em/}/<0,-100>[`;\beta']
\morphism(200,-55)|b|<100,0>[`;\delta]
\efig
\quad\mapsto\quad
\bfig\scalefactor{2}
\morphism/{@{>}@/^1.3em/}/[a`c;f]
\morphism|b|/{@{>}@/_1.3em/}/[a`c;f'']
\morphism(180,50)/{@{>}@/_.1em/}/<0,-100>[`;\beta\cdot\alpha]
\morphism(320,50)|r|/{@{>}@/^.1em/}/<0,-100>[`;\beta'\cdot\alpha']
\morphism(200,0)<100,0>[`;\delta\cdot\gamma]
\efig
\end{equation}
% 
We formalise this as follows.

\subsubsection{Telescopes}
Firstly, we have to define when two cells are \emph{composable}. To
this end we introduce \emph{telescopes}.
Informally, a telescope is a path from a category to one of its
indirect hom-categories. 
Formally, telescopes, $\Tel$, are defined below at the same
time as their \emph{concatenation} onto a category, $\dblplus$, which takes a
telescope to a category, and therefore allows us to put objects into a
telescope:
\[
\AxiomC{$C : \Cat~\Gamma$}
\AxiomC{$n : \Nat$}
\BinaryInfC{$\Tel~C~n : \Set$}
\DisplayProof
\quad
\AxiomC{$t : \Tel~C~n$}
\UnaryInfC{$C \dblplus\, t \,:\,
  \Cat~\Gamma$}
\DisplayProof
\]
Telescopes are like categories except that the base case is an
arbitrary category $C$ rather than $\bullet$\,:
\[\AxiomC{$\phantom{C}$}
\UnaryInfC{$\bullet : \Tel~C~0$}
\DisplayProof
\quad
\AxiomC{$t : \Tel~C~n \quad a, b : \Obj\,(\conctel{C}{t})$}
\UnaryInfC{$ \telsuc{t}{a}{b} : \Tel~C~(n+1)$}
\DisplayProof
\]
%
Here, we call $n$ the \emph{length of $t$}, and we say any $x :
\Obj (\conctel{C}{t})$ to be \emph{of depth $n$}.
%
Note that only the left associative reading of $\dblplus$ makes sense
so expressions like $C \dblplus\, t \dblplus\, u$ are unambiguous.

We say that an object $x : \Obj~(\conctel{C}{t})$ \emph{lies in (the
  telescope) $t$}. When $t$ is relative to a category $C$, $x$ is
called an \emph{object relative to $C$.} 
%
Alternatively, when the path from $C$ to its subcategory is not
important we use the following syntactical shorthand:
\[
\AxiomC{$C : \Cat~\Gamma \quad t : \Tel~C~n$}
\UnaryInfC{$\cat{t}\, : \Cat~\Gamma \qquad \cat{t}\,=\,\conctel{C}{t}$}
\DisplayProof
\]

For example, given the category 
\[\bfig
\morphism(0,1000)/{@{>}@/^2em/}/<1000,0>[a`b;f]
\morphism(0,1000)|b|/{@{>}@/_2em/}/<1000,0>[a`b;g]
\morphism(350,1150)|l|/{@{>}@/_.5em/}/<0,-300>[`;\varphi]
\morphism(650,1150)|r|/{@{>}@/^.5em/}/<0,-300>[`;\gamma]
\efig
\enspace,\]
one has:
\[\cat{\telsuc{\telzero}{\varphi}{\gamma}}~ = ~
\conctel{\homcat{\homcat{\bullet}{a}{b}}{f}{g}}{\telsuc{\telzero}{\varphi}{\gamma}} ~ = ~ \homcat{\homcat{\homcat{\bullet}{a}{b}}{f}{g}}{\varphi}{\gamma}\enspace.\]
%




\subsubsection{Back to composition}
We use telescopes to define syntax for all compositions of an
$\omega$-category. These are defined mutually recursively with
their extensions to telescopes:
%
\[
\AxiomC{$
\alpha : \Obj (t \Downarrow)
\qquad 
\beta : \Obj (u \Downarrow)
$}
\UnaryInfC{$\beta\circ \alpha : \Obj (u \circ t \Downarrow)$}
\DisplayProof
\]
is  a new constructor of $\Obj$ where 
\[
\AxiomC{$t : \Tel~(\homcat{C}{a}{b})~n \qquad u :
  \Tel~(\homcat{C}{b}{c})~n$}
\UnaryInfC{$u \circ  t  : \Tel~(\homcat{C}{a}{c})$}
\DisplayProof
\]
is a function on telescopes defined by cases
\[
{\telzero \circ \telzero \,=\,\telzero\qquad
\telsuc{u}{a'}{b'} \circ \telsuc{t}{a}{b} \,=\, \telsuc{(u
    \circ t) }{a' \circ a}{b' \circ b}}
\]
%
Any $\alpha$ and $\beta$ as above are said to be \emph{composable}.
Note that for a fixed category $C$, $\circ$ always defines the
composition in $C$, called \emph{horizontal} in the 2-categorical
case, which can be applied to all composable $(n+1)$-cells of $C$, where
$n$ is the length of the telescopes $t$ and $u$. To compose cells
``vertically'', one moves to a homcategory. In
2-category theory, horizontal composition is usually denoted $\circ$ or
$\ast$ or is left out, whereas vertical composition by
$\cdot\,$. In our case, we always use $\circ$ and
the level we mean is contained in the (implicit) parameter  $C$. For
example, \eqref{eq:comp1},\eqref{eq:comp2} are both horizontal
compositions where $C = \bullet$, while \eqref{eq:comp4} is a
vertical composition where $C = \homcat{\bullet}{a}{b}$. 



\subsection{Units}\label{sec:units}
Any $n$-cell has $n$ compositions and so it should have $n$ units. We
generate all from a single constructor $\id$ defined as follows:
\[
\AxiomC{$a : \Obj~C$}
\UnaryInfC{$\id~a : \Obj ~\homcat{C}{a}{a}$}
\DisplayProof
\]
%
By iteration we obtain the unit for horizontal composition of $n$-cells:
\[
\AxiomC{$a : \Obj~C$}
\AxiomC{$ n : \Nat$}
\BinaryInfC{$ \idTel{a}{n} : \Tel ~C~n \qquad \id^n a : \Obj~ (\idTel{a}{n} \Downarrow) $}
\DisplayProof
\]
Again, an iterated unit is defined at the same time as its
telescope.
\[
\AxiomC{$\idTel{a}{0} \,=\, \telzero$}
\DisplayProof
\AxiomC{$\idTel{a}{(n+1)}\, = \,
  \telsuc{(\idTel{a}{n})}{\id^n a}{\id^n a}$}
\DisplayProof
\]
\[
\AxiomC{$\id^0 a \, =\, a$}
\DisplayProof
\quad 
\AxiomC{$\id^{(n+1)}a\,=\, \id\, (\id^n a)$}
\DisplayProof
\]




%%% Local Variables: 
%%% mode: latex
%%% TeX-master: "weakomega2"
%%% End: 


\section{Laws}
\label{sec:laws}


%
In a strict $\omega$-category composition and
identities -- structure -- are accompanied by axioms expressing their fundamental
properties. Namely, composition should be \emph{associative}, and it
should satisfy the so-called \emph{interchange law}; identities should
be the \emph{units} of composition. The fact that the axioms are
equations and one can therefore replace equals for equals in
expressions has the pleasant consequence that the
complexity of axioms doesn't increase with dimension. Indeed, the whole theory
for strict $\omega$-categories can be generalised without much difficulty
to categories enriched in an arbitrary monoidal
category \cite{kelly:1982}. However, once 
the equational axioms are replaced by data -- \emph{coherence cells} -- their
complexity rises steeply with dimension. 

This combinatorial complexity of coherence cells has been a major
obstacle in the development of weak $\omega$-categories. This has led
to the development of many diverse approaches to weak
$\omega$-categories,
e.g. \cite{street87:simplexes,batanin98:monoidal-globular,baez:1998,
trimble:1999,penon:1999,leinster:2000,lumsdaine10:weak-o-categories}. Comprehensive
surveys and comparisons can be found in
\cite{leinster:survey,cheng:guidebook}. However the development of
Type Theory has made it possible to express all coherence cells in a
closed form. In this section we start to describe how in Type Theory
all coherence cells can be generated by induction on their depth.


For example, the 1-categorical left-unity law: 
\begin{equation}\label{eq:lambda-f-strictly}
\id_b \circ f \quad = \quad f \enspace,
\end{equation}
 for all $f : a \longrightarrow b$, is replaced in a weak
$\omega$-category by a pair of 2-cells $\lambda_f: \id_b \circ f
\Longrightarrow f $ and $\lambda^{-1}_f : f \Longrightarrow \id_b \circ
f$. A similar law should hold for $\circ$ and higher cells. I.e. it
should also hold in the strict case that for any $\alpha: f
\Longrightarrow f'$: 
\begin{equation}\label{eq:lambda-alpha-strictly}
\id^2_b \circ \alpha \quad = \quad \alpha\enspace,
\end{equation}
where $\id^2_b = \id_{\id_b}$. Note that
\eqref{eq:lambda-alpha-strictly} makes sense because
\eqref{eq:lambda-f-strictly} holds. In the weak case, it is not the
case that the boundary of $\id_b^2\circ\alpha$ is equal to the
boundary of $\alpha$ and it is simply not possible to categorify
\eqref{eq:lambda-alpha-strictly} by introducing a pair of 3-cells
between the left and right side of
\eqref{eq:lambda-alpha-strictly}. However, we can use $\lambda_f$ and
$\lambda^{-1}_f$ to coerce the boundary of the former, $\id_b \circ f$
and $\id_b \circ f'$, to the boundary of the latter, $f$ and 
$f'$, respectively. The following
figure illustrates this idea:
\begin{equation}\label{eq:lambda-alpha-morphism}
\lambda_\alpha \; : \;
\bfig
\morphism/{@{>}@/^1em/}/[a`b;f]
\morphism(500,0)/{@{>}@/^1em/}/[b`b;\id_b]
\morphism/{@{>}@/^3em/}/<1000,0>[a`b;f]
\morphism(500,250)|r|/=>/<0,-100>[`;\lambda^{-1}_f]
\morphism|b|/{@{>}@/_1em/}/[a`b;f']
\morphism(500,0)|b|/{@{>}@/_1em/}/[b`b;\id_b]
\morphism(250,50)/=>/<0,-100>[`;\alpha]
\morphism(750,50)/=>/<0,-100>[`;\id^2_b]
\morphism|b|/{@{>}@/_3em/}/<1000,0>[a`b;f']
\morphism(500,-100)|r|/=>/<0,-100>[`;\lambda_{f'}]
\efig
\quad\Rrightarrow\quad
\bfig\
\morphism/{@{>}@/^2em/}/<500,0>[a`b;f]
\morphism(250,50)/=>/<0,-100>[`;\alpha]
\morphism|b|/{@{>}@/_2em/}/<500,0>[a`b;f']
\efig
\end{equation}
% 
The reader is invited to try to write down the fourth iteration,
i.e. the domain and codomain of $\lambda_\gamma$ for $\gamma : \alpha
\Rrightarrow \alpha' : f \Longrightarrow 'f : a \longrightarrow
a'$. Note that each higher pair of $\lambda$'s can be seen as
expressing the naturality of the preceding lower lambda. 

Similarly one must introduce $\rho$'s to witness the right unit law,
$\chi$'s to witness interchange, and $\alpha$'s to witness
associativity. The example of $\lambda$
has been chosen because of its relative simplicity. 

Moreover, all such coherence cells must satisfy a coherence property
basically saying that any pair of $n$-cells from $d$ to $d'$ involving
only coherence cells and units\footnote{Identity cells can be seen as
  coherence cells witnessing reflexivity of equality.} must have a
mediating n+1-cell connecting $d$ and $d'$. Intuitively, as the
coherence cells $\lambda$, $\rho$, $\alpha$ and $\chi$ we have just
described witness axioms, the higher coherence cells witness their
closure under composition and identity.

\subsection{Formalising left units}
\label{sec:lambdas}

In \eqref{eq:lambda-alpha-morphism} we made the boundaries of the left- and right-hand
sides match by applying the function:
\[
\Phi \quad \equiv \quad (l,l') \quad \mapsto \quad x\quad
\mapsto \quad l' \cdot x \cdot l \] 
%
to $(\lambda^{-1}_f, \lambda_{f'})$ and $\id^2_b \circ \alpha$. The 3-cells
$\lambda_\alpha$ and $\lambda^{-1}_\alpha$ are then introduced as
%
\[\begin{array}{ll}\lambda_\alpha & : \quad
  \Obj~(\homcat{\homcat{\homcat{\bullet}{a}{b}}{f}{f'}}{\Phi\,(\lambda^{-1}_f,\lambda_{f'}) \,
    (\id^2_b \circ \alpha) }{\alpha})\\
\lambda^{-1}_\alpha & : \quad \Obj~(\homcat{\homcat{\homcat{\bullet}{a}{b}}{f}{f'}}{\alpha}{\Phi\,(
 \lambda^{-1}_f,\lambda_{f'})\,(\id^2_b\circ \alpha)})
\end{array}
\enspace.\] 
%
%
These arrows should be natural in 3-cells $\gamma : \alpha \Rrightarrow
\alpha'$. 
In a diagram:
\[
\bfig
\square/<-`>`>`>/<750,500>[ \id_b \circ f ` f  ` \id_b \circ
f' ` f';\lambda^{-1}_f`\id^2_b\,\circ\,\alpha`\alpha`\lambda_{f'}]
\morphism(300,180)/@3{->}/<200,0>[`;\lambda_\alpha]
\morphism(300,320)/@3{<-}/<200,0>[`;\lambda^{-1}_\alpha]
\efig
\qquad
\bfig
\square/<-`>`>`>/<1000,750>[ \lambda_{f'}\ast (\id^2_b \circ \alpha)
\ast \lambda^{-1}_f ` \alpha  ` \lambda_{f'}\ast(\id^2_b \circ
\alpha')\ast \lambda^{-1}_f ` \alpha';\lambda^{-1}_\alpha` \id_{\lambda_{f'}}\ast(\id^3_b\,\circ\,\gamma)\ast \id_{\{\lambda^{-1}_f}`\gamma`\lambda_{\alpha'}]
\morphism(350,300)/@3{->}/<200,0>[`;\lambda_\gamma]
\morphism(350,440)/@3{<-}/<200,0>[`;\lambda^{-1}_\gamma]
\efig
\]
%
Note that going top-left-bottom  around the square one gets
%
\[\Phi\,(\lambda^{-1}_\alpha, \lambda_{\alpha'})\,
(\Phi\,(\lambda^{-1}_f, \lambda_{f'}) \, \gamma))\enspace.\]
%
This is the basic idea of the recursion generating all higher
$\lambda$'s. A similar pattern occurs in the definition of the other
coherence cells.


\subsection{Formalising all coherence cells}\label{sec:formalising-coherence}
To summarise and generalise, we want to introduce
for each $\alpha$ in a telescope $t$ of length $n$ and each $\beta$ in
a telescope $u$ of length $n$ a cell $\Phi\,m \,\alpha \longrightarrow
\beta$ where $m$ is the data necessary to define a function
$\Obj~(\cat{t}) \rightarrow \Obj~(\cat{u})$. We will call such an $m$
a \emph{telescope morphism form $t$ to $u$}; formally $ m : t
\rightrightarrows u$. Then $\Phi$ has type $t \rightrightarrows u
\rightarrow \Obj (\cat t) \rightarrow \Obj(\cat u)$. Formally, we
define telescope morphisms as follows; in mutual recursion to their
application to telescopes and objects in telescopes:
\[
\AxiomC{$t,u : \Tel\, C\, n $}
\UnaryInfC{$t \rightrightarrows u : \Set$}
\DisplayProof
\qquad
\AxiomC{$\phantom{\bullet : \telzero \rightrightarrows \telzero}$}
\UnaryInfC{$\bullet : \telzero \rightrightarrows \telzero$}
\DisplayProof
\qquad
\AxiomC{$m : t \rightrightarrows u \quad \alpha :
  \Obj\,(\telsuc{\cat{u}}{a'}{\appobj{m}{a}}) \quad
  \Obj\,(\telsuc{\cat{u}}{\appobj{m}{b}}{b'})$}
\UnaryInfC{$\telsuc{m}{\alpha}{\beta} : \telsuc{t}{a}{b}
    \rightrightarrows \telsuc{u}{a'}{b'}$}
\DisplayProof
\]
where
\[\AxiomC{$m : t \rightrightarrows u \quad t' : \Tel\, (\cat{t})\,n$}
\UnaryInfC{$\apptel{m}{t'} : \Tel\,(\cat{u})\,n$}
\DisplayProof
\qquad
\AxiomC{$m : t \rightrightarrows u \quad a : \Obj\,(\cat{t} \dblplus {'t})$}
\UnaryInfC{$\appobj{m}{a} : \Obj\,(\cat{u}\dblplus \apptel{m}{t'})$}
\DisplayProof\]
\begin{align*}
\apptel{\telzero}{t}&=t\\
\apptel{\telsuc{m'}{\alpha}{\beta}}{t} &= \telsuc{(\apptel{m'}{t})}{\appobj{m}{\alpha}}{\appobj{m}{\beta}})
\end{align*}
% 
To define $\appobj{}{}$ we need the following auxiliary function,
among others, which extends a telescope relative to a hom-category on
the left.
\[
\AxiomC{$t : \Tel~(\homcat{C}{a}{b})~n$}
\UnaryInfC{$\suctel{a}{b}{t} : \Tel~C ~ (n+1)$}
\DisplayProof
\qquad \text{where} \qquad
\begin{array}{rl}
\suctel{a}{b}{\telzero}& =~ \telsuc{\telzero}{a}{b}\\
\suctel{a}{b}{(\telsuc{t}{c}{d})}&=\; \telsuc{(\suctel{a}{b}{t})}{c}{d}
\end{array}
\]
Note that here $c$ and $d$ don't actually fit into the telescope
$\suctel{a}{b}{t}$ because the latter is definitionally different from
$t$. However, it's straightforward to prove by induction that 
%
\begin{equation}\label{eq:lem-subtel} \cat{t} \quad \equiv \quad \cat{\suctel{a}{b}{t}}\enspace,
\end{equation}
% 
and use the proof to make $c$ and $d$ fit. 
However, in the interest of avoiding syntactical clutter we usually just silently assume
that $c$ and $d$ fit or can be substituted to fit the context.

We are now in the position do define $\appobj{}{}$ as follows:
The base case is trivial:
\[
\appobj{\telzero}{x} \; = \; x
\]
The hom-case follows the pattern outlined in Section \ref{sec:lambdas}. 
\[
\AxiomC{$
\homcat{m'}{\alpha}{\beta} : \homcat{t}{a}{b} \rightrightarrows \homcat{u}{a'}{b'}\qquad t' :
\Tel~(\cat{\homcat{t}{a}{b}}) ~ n\qquad x : \Obj~{(\cat{t'})}
$}
\UnaryInfC{$\appobj{\homcat{m'}{\alpha}{\beta}}{x}\;=\;\id^n\beta
    \,\circ\, (\appobj{m'}{x}) \,\circ \, \id^n\alpha$}
\DisplayProof
\]
%
In summary, $\appobj{m}{x}$ is defined by induction on $m$ where in
each step the length of $m$ decreases by one and the depth of $x$
increases by one. To make the levels match the category of $x$ 
has to be \emph{whiskered} by the morphisms $\alpha$, $\beta$ for $m =
\telsuc{m'}{\alpha}{\beta}$. When $m = \telzero$, the recursion
stops. The meticulous reader will have noticed that the expression
$\appobj{m'}{x}$ above is not well typed as $x$ lives in
$\conctel{\telsuc{t}{a}{b}}{t'}$ and we need an object in
$\conctel{t}{\suctel{a}{b}{t'}}$. But this is easily fixed by
substituting using \eqref{eq:lem-subtel}. Other similar inaccuracies
are dealt with similarly.

Here is an illustration of $m =
\telsuc{\telsuc{\telzero}{\varphi}{\gamma}}{\alpha}{\beta}$, $t' = \telzero$,
$t = \telsuc{\telsuc{\telzero}{a}{b}}{f}{g}$, $u =
\telsuc{\telsuc{\telzero}{a'}{b'}}{f'}{g'}$:
\[\bfig
\node a'(0,0)[a']
\node b'(2000,0)[b']
\node a(666,0)[\appobj{m}{a}]
\node b(1372,0)[\appobj{m}{b}]
\arrow/{@{>}@/^4em/}/[a'`b';f']
\arrow|b|/{@{>}@/_4em/}/[a'`b';g']
\arrow|m|/{@{>}@/^1.5em/}/[a'`a;\varphi]
\arrow|m|/{@{>}@/^1.5em/}/[a`b;\appobj{m}{f}]
\arrow|m|/{@{>}@/^1.5em/}/[b`b';\gamma]
\arrow|m|/{@{>}@/_1.5em/}/[a'`a;\varphi]
\arrow|m|/{@{>}@/_1.5em/}/[a`b;\appobj{m}{g}]
\arrow|m|/{@{>}@/_1.5em/}/[b`b';\gamma]
\morphism(330,50)|r|/=>/<0,-100>[`;\id_\varphi]
\morphism(1705,50)|r|/=>/<0,-100>[`;\id_\gamma]
\morphism(999,50)|r|/=>/<0,-100>[`;\appobj{m}{x}]
\morphism(999,350)|r|/=>/<0,-100>[`;\alpha]
\morphism(999,-250)|r|/=>/<0,-100>[`;\beta]
\efig
\]

Having defined telescope morphisms, it's easy to define $\lambda$'s of all depths
relative to an arbitrary category. All that is needed is a telescope
morphism, $\overrightarrow{\lambda}$,  together with a new
constructor, $\lambda$, of $\Obj$:
\[\AxiomC{$t : \Tel~\homcat{C}{a}{b}~n $}
\UnaryInfC{$\overrightarrow{\lambda}\,t : (\idTel (\id \, b) \, n) \circ t
  \rightrightarrows t$}
\DisplayProof
\qquad
\AxiomC{$t : \Tel~\homcat{C}{a}{b}\, n \quad f : \Obj (\cat{t}) $}
\UnaryInfC{$\lambda\,t\,f :
  \Obj\,(\homcat{(\cat{t})}{\appobj{\overrightarrow{\lambda}t}{(\id^n
      b\,\circ\, f)}}{f})$}
\DisplayProof
\]
where 
\[
\AxiomC{$\overrightarrow{\lambda}\,\bullet ~=~ \bullet
\qquad 
\overrightarrow{\lambda}\,(\telsuc{t'}{a}{b}) ~ = ~
\telsuc{(\overrightarrow{\lambda}\,t')}{\inv{(\lambda\,t\,a)}}{\lambda\,t\,b}$}\DisplayProof
\]
%
Here we could define a pair of constructors $\lambda$ and
$\inv{\lambda}$ for the two opposite directions of $\lambda$. Instead,
as we are interested only in groupoids, we define a generic constructor
$\inv{}$ on all cells of a homcategory:
\[
\AxiomC{$t : \Tel\,(\homcat{C}{a}{b})\,n$}
\AxiomC{$a : \Obj~t$}
\BinaryInfC{$\inv{a} : \Obj\,(\conctel{\homcat{C}{b}{a}}{\inv{t}})$}
\DisplayProof\]
where $\inv{}$ extends recursively to telescopes in the obvious way.



\subsection{Right units, associativity and interchange}
Similarly to $\lambda$'s we define the remaining coherence cells,
i.e. $\rho$'s to witness \emph{right units}, $\alpha$'s to witness
\emph{associativity} of composition and $\chi$'s to witness
interchange. These are defined analogically to $\lambda$'s.

To this end, note that everything in the definition of $\lambda$ is
forced by the type of $\overrightarrow{\lambda}$. In general it is
enough to give, for
each of $\rho$, $\alpha$ of $\chi$,  the type of the telescope
morphism. Just as in the case of $\lambda$, it is in each case just a
``telecopisation'' of the ordinary case.
For right units and associativity we define:
\[
\AxiomC{$t : \Tel\,\homcat{C}{a}{b}\,n$}
\UnaryInfC{$\overrightarrow{\rho}\,t : t \circ (\idTel\,(\id\,a)\,n)  \rightrightarrows t$}
\DisplayProof
\qquad
\AxiomC{$t : \Tel~\homcat{C}{a}{b}~m\quad u :
  \Tel~\homcat{C}{b}{c}~n \quad v : \Tel~\homcat{C}{c}{d}~o$}
\UnaryInfC{$\overrightarrow{\alpha}\,t\,u\,v  : (v \circ u) \circ t
  \rightrightarrows v \circ(u \circ t)
$}
\DisplayProof
\]
%
$\overrightarrow{\chi}$ is in the $\omega$ case a bit more
complicated. In the simple 2-categorical case,
the interchange law states that the two possible ways of composing the
following diagram are equal. 
\[
\bfig
\morphism|m|<500,0>[a`b;f']
\morphism/{@{>}@/^2em/}/<500,0>[a`b;f]
\morphism|b|/{@{>}@/_2em/}/<500,0>[a`b;f'']
\morphism(250,150)|r|/=>/<0,-100>[`;\varphi]
\morphism(250,-50)|r|/=>/<0,-100>[`;\varphi']
%
\morphism(500,0)|m|<500,0>[b`c;g']
\morphism(500,0)/{@{>}@/^2em/}/<500,0>[b`c;g]
\morphism(500,0)|b|/{@{>}@/_2em/}/<500,0>[b`c;g'']
\morphism(750,150)|r|/=>/<0,-100>[`;\gamma]
\morphism(750,-50)|r|/=>/<0,-100>[`;\gamma']
\efig
\]
Formally:
\begin{equation}\label{eq:2cat-interchange}
 (\gamma'\cdot\gamma)\ast(\varphi'\cdot\varphi) \quad = \quad
(\gamma'\ast \varphi')\cdot(\gamma\ast\varphi)
\end{equation}
%
In the $\omega$-case \eqref{eq:2cat-interchange}
remains syntactically the same but we consider each of $\varphi$,
$\varphi'$, $\gamma$ and $\gamma'$ in their telescopes with the generalised
notion of composability.  The following
picture illustrates the idea:
\[
\bfig\scalefactor{2.3}
\morphism/{@{>}@/^3.5em/}/<500,0>[a`b;]
\morphism|b|/{@{>}@/_3.5em/}/<500,0>[a`b;]
\morphism(500,0)/{@{>}@/^3.5em/}/<500,0>[b`c;]
\morphism(500,0)|b|/{@{>}@/_3.5em/}/<500,0>[b`c;]
\place(430,0)[\cdots]
\place(930,0)[\cdots]
\morphism(150,130)/{@{>}@/_1em/}/<0,-260>[`;]
\morphism(340,130)/{@{>}@/^1em/}/<0,-260>[`;]
%%
\place(60,35)[_{u_1}]
\place(50,0)[\cdots]
\place(560,35)[_{u_2}] 
\place(550,0)[\cdots]
\morphism(650,130)/{@{>}@/_1em/}/<0,-260>[`;]
\morphism(840,130)/{@{>}@/^1em/}/<0,-260>[`;]
%%
\morphism(150,95)/{@{>}@/^.5em/}/<200,0>[`;a_1]
\morphism(150,-95)|b|/{@{>}@/_.5em/}/<200,0>[`;c_1]
\morphism(150,0)|m|<200,0>[`;b_1]
\place(155,55)[\cdots]
\place(155,-55)[\cdots]
\place(345,55)[\cdots]
\place(345,-55)[\cdots]
\morphism(200,100)/{@{>}@/_.2em/}/<0,-90>[`;]
\morphism(300,100)/{@{>}@/^.2em/}/<0,-90>[`;]
\morphism(200,-10)/{@{>}@/_.2em/}/<0,-90>[`;]
\morphism(300,-10)/{@{>}@/^.2em/}/<0,-90>[`;]
\place(165,85)[{_{t_{11}}}]
\place(165,-15)[{_{t_{12}}}]
\place(665,85)[{_{t_{21}}}]
\place(665,-15)[{_{t_{22}}}]
\place(655,55)[\cdots]
\place(655,-55)[\cdots]
\place(845,55)[\cdots]
\place(845,-55)[\cdots]
\morphism(650,95)/{@{>}@/^.5em/}/<200,0>[`;a_2]
\morphism(650,-95)|b|/{@{>}@/_.5em/}/<200,0>[`;c_2]
\morphism(650,0)|m|<200,0>[`;b_2]
\morphism(700,100)/{@{>}@/_.2em/}/<0,-90>[`;]
\morphism(800,100)/{@{>}@/^.2em/}/<0,-90>[`;]
\morphism(700,-10)/{@{>}@/_.2em/}/<0,-90>[`;]
\morphism(800,-10)/{@{>}@/^.2em/}/<0,-90>[`;]
%
% \morphism(225,55)<50,0>[`;\alpha_{11}]
% \morphism(225,-55)<50,0>[`;\alpha_{12}]
% \morphism(725,55)<50,0>[`;\alpha_{21}]
% \morphism(725,-55)<50,0>[`;\alpha_{22}]
\efig
\]
%
Where $\cdots$ indicate telescopes of arbitrary depth although $u_1$
and $u_2$ have to be of the same length; and $t_{ij}, \; i,j \in
\{1,2\}$ have to be of the same length. 
In this situation, it is possible to form both the composition $(t_{22}\circ
t_{21})\circ(t_{12}\circ t_{11})$ and also $(t_{22}\circ
t_{12})\circ(t_{21}\circ t_{11})$. A telescope morphism from the
former to the latter telescope induces a coherence cell for
interchange. This is formalised as follows:
\[
\AxiomC{$u_1 : \Tel~\homcat{\mathcal{C}}{a}{b}~n\quad u_2 : \Tel~\homcat{\mathcal{C}}{b}{c}~n\quad
t_{11} :  \Tel~(\conctel{\homcat{C}{a}{b}}{u_1}) ~ m$}
\noLine
\UnaryInfC{$t_{12} : \Tel~(\conctel{\homcat{C}{a}{b}}{u_1}) ~ m\quad t_{21} : \Tel~(\conctel{\homcat{C}{b}{c}}{u_2}) ~ m 
\quad t_{22} : \Tel~(\conctel{\homcat{C}{b}{c}}{u_2})~ m$}
% 
\UnaryInfC{$\chi\,{t_{11}}\,{t_{12}}\,{t_{21}}\,{t_{22}} \;:\;(t_{22}\circ t_{21})\circ(t_{12}\circ t_{11}) \rightrightarrows (t_{22}\circ
    t_{12})\circ(t_{21}\circ t_{11})$}
\DisplayProof
\]



%%% Local Variables: 
%%% mode: latex
%%% TeX-master: "weakomega2"
%%% End: 



\section{Coherence}
\label{sec:coherence}

\begin{quote}
  \begin{itemize}
  \item hollow (~ closed in a sense)
  \end{itemize}
\end{quote}

\subsection{Coherence cells between coherence cells}
Finally we add all coherence cells between parallel coherence
cells and their compositions. To this end we define a predicate
$\hollow$ on objects:
\[
\AxiomC{$x :  \Obj~C$}
\UnaryInfC{$\hollow~x~:~\Set$}
\DisplayProof
\]
%
Such that all identities, $\alpha$'s, $\rho$'s, $\lambda$'s and
$\chi$'s are hollow, and all weakenings and compositions of
hollow cells are hollow. Then we introduce a constructor of $\Obj$:
\[
\AxiomC{$f ~ g : \Obj~\homcat{C}{a}{b}$}
\AxiomC{$p : \hollow~f$}
\AxiomC{$q : \hollow~g$}
\TrinaryInfC{$\mathsf{coh}~f~g~p~q :
  \Obj~\homcat{\homcat{C}{a}{b}}{f}{g}$}
\DisplayProof
\]
%
And we define $\hollow~(\mathsf{coh}~f~g~p~q) = \top$.




%%% Local Variables: 
%%% mode: latex
%%% TeX-master: "weakomega2"
%%% End: 



\section{Id$\omega$ is an $\omega$-Groupoid \textit{(txa,oxr)}}
\label{sec:idw}

\begin{quote}
  \begin{itemize}
  \item Idw is an w-Groupoid (txa,oxr)
  \end{itemize}
\end{quote}
%%% Local Variables: 
%%% mode: latex
%%% TeX-master: "weakomega2"
%%% End: 


\section{Conclusions and Further Work}
\label{sec:conclusions}

We have presented a novel approach to defining weak $\omega$-groupoids
which is based on ideas from Type Theory. A central idea is to define
the syntax of weak $\omega$-groupoids which is where Type Theory has
its greatest strength. Indeed, we have formalized most of the material
presented here in Agda \cite{agda}.  We believe that our approach is
quite natural, in a way naive, since our approach to formalizing coherence
cells seems a natural generalisation of the corresponding first
order laws.

The current formalisation is still
quite big and we hope to find ways to simplify it. One interesting
idea may be to use the recent approach to define opetopes based on
dependent polynomial functors (i.e. indexed containers) \cite{opetopes},
which has a very type-theoretic flavour.  

We would like to use our framework to provide a formalisation of a
variation of the results by in
\cite{lumsdaine10:weak-o-categories,berg08:types-are} by showing that
$\Idw$ is a weak $\omega$-groupoid. This is slightly different form
their results because we are working insinde Type Theory instead of on
a metalevel.

The main challenge ahead is to formalize the notion of an
$\omega$-groupoid model of Type Theory. Once this has been done we
will be able to eliminate the univalence axiom and provide a
computational interpretation of this principle.


%%% Local Variables: 
%%% mode: latex
%%% TeX-master: "weakomega2"
%%% End: 



  

\section*{Acknowledgment}


The authors would like to thank...

Peter Lumsdaine, Darin Morrison


\bibliographystyle{alpha}
%\bibliographystyle{IEEEtran}
\bibliography{IEEEabrv,weakomega2.bib}
%\bibliography{weakomega2}

\end{document}




