
\section{Laws}
\label{sec:laws}

\begin{quote}
  \begin{itemize}
  \item telescope morphisms

  \item inverse

  \item lambda,rho,alpha,interchange,inverse-laws

  \end{itemize}
\end{quote}

\subsection{Left-unit coherence cells}
%
In a strict $\omega$-category units are accompanied by laws making
them unital with respect to composition. In the weak setting such laws
are replaced by coherence cells, which we discuss next.

With the development of the previous two sections we can express
compositions such as:
\[
\id\,b \circ f \qquad \id^2\,b \circ \alpha\qquad \id\,f'\circ \alpha
\]
pictured respectively from left to right as
\[
\bfig
\morphism[a`b;f]
\morphism(500,0)<250,0>[b`b;\id\,f]
\efig
\qquad
\bfig
\morphism/{@{>}@/^1em/}/[a`b;f]
\morphism|b|/{@{>}@/_1em/}/[a`b;f']
\morphism(250,50)<0,-100>[`;\alpha]
\morphism(500,0)/{@{>}@/^1em/}/[b`b;\id\,b]
\morphism(500,0)|b|/{@{>}@/_1em/}/[b`b;\id\,b]
\morphism(750,50)<0,-100>[`;\id^2 b]
\efig
\]\[
\bfig\scalefactor{1.2}
\morphism/{@{>}@/^1.7em/}/[a`b;f]
\morphism|-|[a`b;f']
\morphism(250,110)<0,-80>[`;\alpha]
%
\morphism|b|/{@{>}@/_1.7em/}/[a`b;f']
\morphism(250,-20)<0,-80>[`;\id\,f']
\efig
\enspace.\]
We want to define the cells that connect these cells to the cells:
\[
\bfig
\morphism[a`b;f]
\efig
\qquad
\bfig
\morphism/{@{>}@/^1em/}/[a`b;f]
\morphism|b|/{@{>}@/_1em/}/[a`b;f']
\morphism(250,50)<0,-100>[`;\alpha]
\efig
\qquad
\bfig
\morphism/{@{>}@/^1em/}/[a`b;f]
\morphism|b|/{@{>}@/_1em/}/[a`b;f']
\morphism(250,50)<0,-100>[`;\alpha]
\efig
\]

Let us analyse the situation at hand.

Firstly, note that the right-most case is just the left-most case in
the category $\homcat{\bullet}{a}{b}$ so from now on we assume
everything is taking place in an arbitrary \emph{base category},
$\cC$.

For the left-most case, assuming $a$ and $b$ are 0-cells of
  $\cC$, we simply introduce the syntax for a 2-cell $\lambda_f$ of $\cC$:
\[\bfig
\qtriangle[a`b`b;f`f`\id\,b]
\morphism(400,400)/<=/<-100,-100>[`;\lambda_f]
\efig\]

Having defined a 2-cell $\lambda_f$ for
  every 1-cell $f$, one can define a 3-cell $\lambda_\alpha$ for every
  2-cell $\alpha : f \Rightarrow f'$ as going from the left-hand side
  to the right-hand side below:
\[\bfig
\morphism/{@{>}@/^1em/}/[a`b;f]
\morphism(500,0)/{@{>}@/^1em/}/[b`b;\id\,f]
\morphism/{@{>}@/^3em/}/<1000,0>[a`b;f]
\morphism(500,250)/=>/<0,-100>[`;\lambda_f]
\morphism|b|/{@{>}@/_1em/}/[a`b;f']
\morphism(500,0)|b|/{@{>}@/_1em/}/[b`b;\id\,b]
\morphism(250,50)/=>/<0,-100>[`;\alpha]
\morphism(750,50)/=>/<0,-100>[`;\id^2 b]
\efig
\quad\Rrightarrow\quad
\bfig
\morphism/{@{>}@/^3em/}/<1000,0>[a`b;f]
\morphism|m|/{@{>}@/^1.6em/}/<1000,0>[a`b;f']
\morphism(500,290)/=>/<0,-100>[`;\alpha]
\morphism|b|/{@{>}@/_1em/}/[a`b;f']
\morphism(500,0)|b|/{@{>}@/_1em/}/[b`b;\id\,b]
\morphism(750,90)/=>/<0,-100>[`;\lambda_{f'}]
\efig
\]
Note however that because we are not working in a strict setting
the pasting together of the left-hand side is not unique. We pick a
particular one:
\[
\lambda_\alpha ~:~ ((\id^2 b) \circ \alpha) \cdot \lambda_f \Rrightarrow \lambda_{f'}\cdot
\alpha\enspace,
\]
where $\circ$ denotes composition in $\cC$ and $\cdot$ denotes
composition in $\homcat{\cC}{a}{b}$. In $\homcat{\cC}{a}{b}$ this is
the diagonal filler in the square
\begin{equation}\label{eq:lambda-natur}
\bfig
\Square[f`(\id\,b) \circ f` f'`(\id\,b)\circ f';\lambda_f`\alpha`(\id^2
b) \circ \alpha`\lambda_{f'}]
\morphism(500,300)/=>/<-200,-150>[`;\lambda_\alpha]
\efig
\end{equation}

This diagram expresses the naturality of the assignment $f \mapsto
\lambda_f$ with respect to cells $\alpha : f \Rightarrow f'$ 
witnessed by $\lambda_\alpha$. 
%

When we put:
\[
\AxiomC{$t : \Tel~\homcat{C}{a}{b}~n$}
\AxiomC{$x : \Obj~(t\Downarrow)$}
\BinaryInfC{$\dom\lambda_0~x~=~x \qquad
  \cod\lambda_0~x~=~\id^n(\id~b)\circ x$}
\DisplayProof
\]
\[
\AxiomC{$t : \Tel~\homcat{C}{a}{b}~n$}
\AxiomC{$x : \Obj~(t\Downarrow)$}
\BinaryInfC{$\lambda_n x : \Obj~(\telsuc{t}{\dom\lambda_n\, x
  }{\cod\lambda_n\,x}\Downarrow)$}
\DisplayProof
\]
then \eqref{eq:lambda-natur} can be rewritten as follows:
\begin{equation}\label{eq:lambda-natur2}
\bfig
\Square[\dom\lambda_0\,f`\cod\lambda_0\,f`\dom\lambda_0~f'`\cod\lambda_0~f';
\lambda_0 f`\dom\lambda_0~\alpha`\cod\lambda_0~\alpha`\lambda_0 f']
\morphism(500,300)/=>/<-200,-150>[`;\lambda_1 \alpha]
\efig\enspace,
\end{equation}
where necessarily 
\[
\AxiomC{$\dom\lambda_1\,\alpha = \cod\lambda_0\,\alpha\cdot
  \lambda_0\,f$}
\DisplayProof
\quad \AxiomC{$\cod\lambda_1\,\alpha = \lambda_0\, f'\cdot
  \dom\lambda_0\,\alpha$}
\DisplayProof
\]
Here $\cdot$ denotes vertical composition of 2-cells and $\circ$ is,
as before, horizontal composition.

%
To generalise and iterate \eqref{eq:lambda-natur2} we
introduce an $n+1$ cell $\quad\lambda~t~x\quad$ for each telescope
$\quad t :\Tel~\homcat{C}{a}{b}~n\quad$ and $\quad x :
\Obj~(t\Downarrow)\quad$. At the same time we introduce functions
$\dom\lambda$ and $\cod\lambda$ taking a telescope $t$, a telescope
$u$ relative to $t$, formally $u : \Tel ~ (t \Downarrow)~ m$ and a
cell $x : \Obj (u\Downarrow)$ to a new telescope and a cell in it so
that the following makes sense:
\begin{equation}\label{eq:lambda-natur3}
\bfig
\Square|ammb|[\dom\lambda~t~\telzero{t\Downarrow}~x`
\cod\lambda~t~\telzero{t\Downarrow}~x`
\dom\lambda~t~\telzero{t\Downarrow}~x'`
\cod\lambda~t~\telzero{t\Downarrow}~x';
\lambda~t~x`
\dom\lambda~t~\telsuc{\telzero{t\Downarrow}}{f}{f'}~\alpha`
\cod\lambda~t~\telsuc{\telzero{t\Downarrow}}{f}{f'}~\alpha`
\lambda~t~x']
\morphism(750,300)/=>/<-200,-150>[`;\lambda_\alpha]
\efig
\end{equation}
%
Formally, we have the following definitions which generate all
$\quad\lambda ~t~x\quad$'s for all telescopes $t$ and objects $x$ in
them.
%
%
\begin{enumerate}
\item $\Obj$ receives a new constructor:
\[
\AxiomC{$t : \Tel~\homcat{\mathcal{C}}{a}{b}~n$}
\AxiomC{$f : \Obj~(t \Downarrow)$}
\BinaryInfC{$\lambda~t~f : \Obj~
  (\lambda\Tel~\telsuc{t}{f}{f}\Downarrow)$}
\DisplayProof
\]
%
%
\item $\lambda\Tel$ is a function on telescopes:
\[
\AxiomC{$t : \Tel~\homcat{\mathcal{C}}{a}{b}~n$}
\UnaryInfC{$\lambda\Tel~t : \Tel~\homcat{\mathcal{C}}{a}{b}~n$}
\DisplayProof
\]
\begin{equation}\label{eq:lambda-tel-def}
  \begin{array}{l}
 \lambda\Tel~\telzero{C} ~= ~ \telzero{C}\\   
 \lambda\Tel~(\telsuc{t}{f}{f'})~=~\telsuc{(\lambda\Tel~t)}{\dom\lambda~t~\telzero{t\Downarrow}~f}{\cod\lambda~t~\telzero{t\Downarrow}~f'}
  \end{array}
\end{equation}
%
%
\item 
Functions $\dom\lambda$ and $\cod\lambda$ used above have the
following types:
\[
\AxiomC{$t : \Tel~\homcat{\mathcal{C}}{a}{b}~n$}
\AxiomC{$u : \Tel~(t \Downarrow)~m$}
\AxiomC{$x : \Obj~u\Downarrow$}
\TrinaryInfC{$\dom\lambda~t~u~x : \Obj~(\homcat{C}{a}{b} \dblplus
  ~ \lambda\Tel~t~ \dblplus~\dom\lambda\Tel~t~u)$}
\DisplayProof
\]
\[
\AxiomC{$t : \Tel~\homcat{\mathcal{C}}{a}{b}~n$}
\AxiomC{$u : \Tel~(t \Downarrow)~m$}
\AxiomC{$x : \Obj~u\Downarrow$}
\TrinaryInfC{$\cod\lambda~t~u~x : \Obj~(\homcat{C}{a}{b} \dblplus
  ~ \lambda\Tel~t~ \dblplus~\cod\lambda\Tel~t~u)$}
\DisplayProof
\]
%
We omit the definitions, which in full detail are a bit technical. Just
note that the base case, $n= 0$ is defined as above for
$\dom\lambda_0$ and $\cod\lambda_0$ respectively, and in the induction
step $\dom\lambda~\{n = k+1\}$ is defined in terms $\cod\lambda~\{n =
k\}$ and $\lambda~\{n = k\}$, and similarly $\cod\lambda~\{n=k+1\}$ is defined
in terms of $\dom\lambda~\{n=k\}$ and $\lambda~\{n=k\}$ of the smaller dimension.

\end{enumerate}

\subsection{Right units, associativity and interchange}
Similarly to $\lambda$'s we define the remaining coherence cells,
i.e. $\rho$'s to witness \emph{right units}, $\alpha$'s to witness
\emph{associativity} of composition and $\chi$'s to witness
interchange. These are defined analogically to $\lambda$'s. Below we present
only the top-level constructor forms. 
\[
\AxiomC{$t : \Tel~\homcat{\mathcal{C}}{a}{b}~n$}
\AxiomC{$f : \Obj~(t \Downarrow)$}
\BinaryInfC{$\rho~t~f : \Obj~
  (\rho\Tel~\telsuc{t}{f}{f})\Downarrow$}
\DisplayProof
\]
\smallskip
\[
\AxiomC{$t : \Tel~\homcat{\mathcal{C}}{a}{b}~n$}
\noLine
\UnaryInfC{$f : \Obj~t \Downarrow$}
\AxiomC{$u : \Tel~\homcat{\mathcal{C}}{b}{c}~n$}
\noLine
\UnaryInfC{$g : \Obj~u \Downarrow$}
\AxiomC{$v : \Tel~\homcat{\mathcal{C}}{c}{d}~n$}
\noLine
\UnaryInfC{$h : \Obj~v \Downarrow$}
\TrinaryInfC{$\alpha~t~u~v~f~g~h : \Obj~(\alpha\Tel~\telsuc{t}{f}{f}~\telsuc{u}{g}{g}~\telsuc{v}{h}{h})\Downarrow$}
\DisplayProof
\]
\smallskip
\[
\AxiomC{$u_1 : \Tel~\homcat{\mathcal{C}}{a}{b}~n$}
\AxiomC{$u_2 : \Tel~\homcat{\mathcal{C}}{b}{c}~n$}
\noLine
\BinaryInfC{$t_{11} : \Tel~\telsuc{u_1}{a_1}{b_1}\Downarrow ~ m \qquad
t_{12} : \Tel~\telsuc{u_1}{b_1}{c_1}\Downarrow ~ m$}
\noLine
\UnaryInfC{$t_{21} : \Tel~\telsuc{u_2}{a_2}{b_2}\Downarrow ~ m \qquad
  t_{22} : \Tel~\telsuc{u_2}{b_2}{c_2}\Downarrow ~ m$}
\noLine
\UnaryInfC{$\alpha_{11} : \Obj~t_{11}\Downarrow\qquad 
\alpha_{12} : \Obj~t_{12}\Downarrow
\qquad
\alpha_{21} : \Obj~t_{21}\Downarrow
\quad
\alpha_{22} : \Obj~t_{22}\Downarrow$}
\UnaryInfC{$\chi~\alpha_{11}~\alpha_{12}~\alpha_{21}~\alpha_{22}: \Obj~(\chi\Tel~t_{11}~t_{12}~t_{21}~t_{22})\Downarrow$}
\DisplayProof
\]

We also include each of $\lambda$, $\rho$, $\alpha$ and $\chi$ in
their reversed forms, denoted $\lambda^-$, $\rho^-$, $\alpha^-$ and
$\chi^-$ respectively, as each of them should be equivalences. At this
point, the only thing we have done is postulating two cells going in
the opposite directions. The following section introduces the
necessary syntax to make them true \emph{weak equivalences}.



%%% Local Variables: 
%%% mode: latex
%%% TeX-master: "weakomega2"
%%% End: 
