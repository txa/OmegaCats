\documentclass[a4paper]{article}

\usepackage{latexsym, amsmath, amssymb}
\usepackage{amsthm}
\RequirePackage{mathrsfs}
\RequirePackage{stmaryrd} %boxes
\usepackage{diagxy}
\usepackage{bussproofs}

\newtheorem{definition}{Definition}
\newtheorem*{remark}{Remark}

\newcommand{\glob}[2]{(\hspace{-3pt}\big({#1},{#2}\big)\hspace{-3pt})}
\newcommand{\ditto}{\mathrm{- " -}}
\newcommand{\mnote}[1]{\marginpar{\footnotesize{#1}}}
\newcommand{\cell}{\mathsf{cell}}
\newcommand{\obj}{\mathsf{obj}}
\newcommand{\Set}{\mathsf{Set}}
\newcommand{\Glob}{\mathsf{Glob}}
\newcommand{\Nat}{\mathbb{N}}
\newcommand{\Con}{\mathsf{Con}}
\newcommand{\Cat}{\mathsf{Cat}}
\newcommand{\Obj}{\mathsf{Obj}}
\newcommand{\Tele}{\mathsf{Tele}}
\newcommand{\suc}{\mathsf{suc}}
\newcommand{\dom}{\mathsf{dom}}
\newcommand{\cod}{\mathsf{cod}}
\newcommand{\sym}[1]{{#1}^{-1}}
\newcommand{\refl}[1]{\id~{#1}}
\newcommand{\trans}[2]{#1 \text{;} #2}
\newcommand{\domrm}{\mathrm{dom}}
\newcommand{\codrm}{\mathrm{cod}}
%\newcommand{\meets}{\mathsf{meets}}
\newcommand{\meets}{\between}
\newcommand{\zeromeets}{\mathsf{zero}}
\newcommand{\sucmeets}[1]{\mathsf{suc}}
\newcommand{\telezero}[1]{\langle{#1}\rangle}
%\newcommand{\telezero}[1]{\diamond}
\newcommand{\telesuc}[3]{#1\llbracket {#2},{#3}\rrbracket}
\newcommand{\cat}[1]{{#1}\Downarrow}
\newcommand{\homcat}[3]{{#1}[#2,#3]}
\newcommand{\budoteq}{\bullet_{\doteq}}
\newcommand{\homdoteqcat}[3]{{#1}[#2,#3]_{\doteq}}
\newcommand{\intpr}[1]{\llbracket #1 \rrbracket}
\newcommand{\id}{\mathsf{id}}
\newcommand{\IdCat}[2]{\mathsf{IdCat}\,{#1}\,{#2}}
\newcommand{\ItId}{\mathsf{ItId}}
\newcommand{\depth}{\mathsf{depth}}
\newcommand{\preptele}[3]{\llbracket #1,#2\rrbracket #3}
\newenvironment{ondrej}{\begin{quote}\footnotesize \textbf{Ondrej:}}{\normalsize\end{quote}}
\newcommand{\substobj}[1]{\ulcorner #1 \urcorner}
\newcommand{\Iddd}[2]{{#1}\,\equiv\,{#2}}
\newcommand{\coh}{\mathsf{coh}}
\newcommand{\cohCat}{\mathsf{cohCat}}
\newcommand{\dblline}{}
\newcommand{\emptycon}{\varepsilon}
\newcommand{\var}{\mathsf{var}}
\newcommand{\wk}{\mathsf{wk}}

\renewcommand{\to}{\longrightarrow}

\begin{document}
\begin{center}\Large{A syntactical approach to weak $\omega$-categories}
\end{center}
\begin{center}
Thorsten Altenkirch and Ond\v rej Ryp\'a\v cek  
\end{center}

\section{Syntax of weak $\omega$-categories}
\label{sec:syntax}
\subsection{Syntactical Globular Sets}
\label{sec:framework}
%
%
We start introducing syntax for globular sets as follows: 
\[
\mathbf{data}\;
\AxiomC{$\mathsf{Con} : \mathsf{Set}$}
\DisplayProof
\;
\mathbf{where}
\;
\AxiomC{\mathstrut}
\dblline
\UnaryInfC{$\varepsilon : \mathsf{Con}$}
\DisplayProof
\;;\;
\AxiomC{$\Gamma : \mathsf{Con}$}
\AxiomC{$C : \mathsf{Cat}~\Gamma$}
\dblline
\BinaryInfC{$(\Gamma , C) : \mathsf{Con}$}
\DisplayProof
\]
\[
\mathbf{data}\;
\AxiomC{$\Gamma : \mathsf{Con}$}
\UnaryInfC{$\mathsf{Cat}~\Gamma : \mathsf{Set}$}
\DisplayProof
\;\mathbf{where}\;
%\AxiomC{$\{\Gamma : \mathsf{Con} \}$}
\AxiomC{$\phantom{\Gamma}$}
\dblline
\UnaryInfC{$\bullet : \mathsf{Cat}~\Gamma$}
\DisplayProof
\;;\;
%\AxiomC{$\{\Gamma\}$}
\AxiomC{$C : \mathsf{Cat}~\Gamma$}
\AxiomC{$a~b : \mathsf{Obj}~C$}
\dblline
\BinaryInfC{$C[ \,a\,,\,b\,] : \mathsf{Cat}~\Gamma$}
\DisplayProof
\]
\[
\mathbf{data}\;
%\AxiomC{$\{\Gamma\}$}
\AxiomC{$C : \mathsf{Cat}~\Gamma$}
\UnaryInfC{$\mathsf{Obj}~C:\mathsf{Set}$}
\DisplayProof
\;
\mathbf{where}
\;\cdots
\]
%
We introduce mutually recursively \emph{contexts}, $\mathsf{Con}$,
\emph{categories}, $\Cat$, 
and \emph{objects}, $\Obj$. Contexts are telescopes of categories;
\emph{categories} in context are built inductively from a base category,
$\bullet$, by formation of homcategories, $\homcat{C}{a}{b}$.
%
We omit the constructors for $\Obj$ for the moment. Their development
is the subject of the rest of the text.  
% We use double horizontal
% lines to introduce constructors of datatypes. 
We leave out as many assumptions as can be inferred from the
context. For example, the assumption $\Gamma : \Con$ can be inferred
in the rule for $\bullet$ introduction. In such cases when we need to
\emph{name} implicit assumptions, we use curly brackets. We also leave
out type annotations where inferable.

% In summary, expressions
% for categories are of the form 
% \[
% \bullet [~a_0~,~b_0~]\cdots[~a_k~,~b_k~]
% \]
% where each $a_i$, $b_i$ are objects in category $\bullet
% [~a_0~,~b_0~]\cdots[~a_{i-1}~,~b_{i-1}~]$. 

%
The role of contexts is to introduce irreducible objects --
variables. Using De Bruijn indexing we introduce nameless variables in
contexts as follows:
\[
\mathbf{data}\;
%\AxiomC{$\{\Gamma\}$}
\AxiomC{$C : \mathsf{Cat}~\Gamma$}
\UnaryInfC{$\mathsf{Var}~C : \mathsf{Set}$}
\DisplayProof
\;\mathbf{where}\;
\AxiomC{$\phantom{I}$}
%\AxiomC{$\\{\Gamma\}$}
%\AxiomC{$\{C : \Cat~\Gamma\}$}
\dblline
\UnaryInfC{$\mathsf{vz}:\mathsf{Var}~(\mathsf{wk}~C)$}
\DisplayProof
\;;
\]
\[
%\AxiomC{$\{\Gamma\}$}
\AxiomC{$v : \mathsf{Var}~C$}
\dblline
\UnaryInfC{$\mathsf{vs}~v : \mathsf{Var}~(\mathsf{wk}~C~D)$}
\DisplayProof
\]
where $\mathsf{wk}$ is weakening defined for categories by
induction on the structure in the obvious way: 
\[
%\AxiomC{$\{\Gamma\}$}
\AxiomC{$C~D : \mathsf{Cat}~\Gamma$}
\UnaryInfC{$\mathsf{wk}~C~D : \mathsf{Cat}~(\Gamma,D)\qquad\mathsf{wk}~\bullet~D\,=\,\bullet$}
\DisplayProof
\]
\[
\AxiomC{$C~D : \mathsf{Cat}~\Gamma$}
\UnaryInfC{$\mathsf{wk}~(\homcat{C}{a}{b})~D\,=\,(\mathsf{wk}~C~D)[\mathsf{wk}~a~D,\mathsf{wk}~b~D]$}
\DisplayProof
\]
% \mathsf{wk} : \{\Gamma\}()
% \rightarrow 
% \enspace,\]
%
from weakening for objects:
\[
%\AxiomC{$\{\Gamma\}$}
\AxiomC{$\{C:\Cat~\Gamma\}$}
\AxiomC{$x : \mathsf{Obj}~C$}
\AxiomC{$D:\Cat~\Gamma$}
\dblline
\TrinaryInfC{$\mathsf{wk}~x~D:\mathsf{Obj}~(\mathsf{wk}~C~D)$}
\DisplayProof
\enspace,
\]
which is defined as a data constructor of $\mathsf{Obj}$. 

Variables become objects by the following data constructor of $\Obj$:
\[
%\AxiomC{$\{C\}$}
\AxiomC{$v : \mathsf{Var}~C$}
\dblline
\UnaryInfC{$\mathsf{var}~v : \mathsf{Obj}~C$}
\DisplayProof
\]
%
This completes the definition of globular sets with variables. To
explicitly define a globular set out of the syntactical presentation
we define globular sets coniductively as thus:

% where each
%   $G_n$ is a set in Voevodsky's sense \cite{}. \mnote{I want a proper
%     Set and the only way to say it in Type Theory is by
%     contractibility.} Formally, we define 
% $\mathsf{Glob}$ by the following \emph{coinductive}
% definition:
% % , where we use Altenkirch's \emph{delay}, $\sharp$, and
% % \emph{force}, $\flat$, on coinductive sets $S^\infty$
% %\cite{}:
\[
\AxiomC{$\{\Gamma : \Con \}$}
\UnaryInfC{$\mathsf{Glob}:\Set$}
\DisplayProof
\quad
\AxiomC{$ O : \Set$}
\AxiomC{$ H : ( a ~ b : \Set) \rightarrow \mathsf{Glob}$}
% \noLine
% \BinaryInfC{$C : (a ~ b : O) \rightarrow
%   \Sigma(p : a \equiv b)((q : a \equiv b) \rightarrow p \equiv q)$}
\BinaryInfC{$\glob{O}{H} : \mathsf{Glob}$}
\DisplayProof
\]
for which we introduce projections:
\[
\mathsf{obj}~\glob{x}{y} = x \qquad \mathsf{hom}~\glob{x}{y} = y 
\]
% Note that axiom $C$ in the definition of $\mathsf{glob}$ is the
% statement that propositional equality in $O$ is contractible.
\noindent
Then for each $\Gamma$ and  $C : \Cat~\Gamma$, 
\[\mathsf{glob}~C =
\glob{\Obj~C}{\lambda~x~y.\mathsf{glob}~\homcat{C}{x}{y}}\]
%
Finally we define the globular set of a context $\Gamma$ as
\begin{equation}\label{eq:glob-of-gamma}
\mathsf{glob}~\Gamma~=~\mathsf{glob}~(\bullet~\{\Gamma\})
\end{equation}
%
% A \emph{globular set morphism} $\glob{O}{H} \to \glob{O'}{H'}$ is an 
% \[f : O \to O'\]
% together with a collection:
% \[h ~ x ~ y : H ~ x ~ y \to H'~(f~x)~(f~y)\enspace.\]


We will need the following definition:
\[
%\AxiomC{$\{\Gamma\}$}
\AxiomC{$C : \Cat~\Gamma$}
\UnaryInfC{$\mathsf{depth}~C : \Nat$}
\DisplayProof
\quad
\AxiomC{$\mathstrut$}
\UnaryInfC{$\mathsf{depth}~\bullet\,=\,0$}
\DisplayProof
\quad
%\AxiomC{$C : \Cat~\Gamma$}
%\AxiomC{$a~b : \Obj~C$}
\AxiomC{$\mathstrut$}
\UnaryInfC{$\mathsf{depth}~(C[a,b])~=~(\mathsf{depth}~C) + 1$}
\DisplayProof
\enspace,\]
where $\Nat$ is the $\Set$ of natural numbers with constructors $0$ and $+1$.
%
% &\mathsf{depth}:(\mathsf{Cat}~\Gamma)\rightarrow \Nat\\
% &\mathsf{depth}~\bullet = 0\\
% &\mathsf{depth}~(C[a,b]) = (\mathsf{depth}~C) + 1
% \end{align*}
%
Then any $x : \Obj~C$ such that $\mathsf{depth}~C \equiv n$, where
$\equiv$ denotes propositional equality, is called an \emph{$n$-cell}.
%
We use the usual arrow notation for categories and objects. For
instance, $\bullet[a,b]$, $\bullet[a,b][f,g]$ and $\alpha :
\Obj~(\bullet[a,b][f,g])$ are pictured respectively as follows:
\[\bfig
\morphism/{}/<300,0>[a`b;]
\efig
\quad\qquad 
\bfig
\morphism/{@{>}@/^1em/}/[a`b;f]
\morphism|b|/{@{>}@/_1em/}/[a`b;g]
\efig
\qquad 
\bfig
\morphism/{@{>}@/^1em/}/[a`b;f]
\morphism|b|/{@{>}@/_1em/}/[a`b;g]
\morphism(250,80)|r|<0,-140>[`;\alpha]
\efig
\]
%
We also write, as usual, $x : a_n\longrightarrow b_n : \cdots
: a_0 \longrightarrow b_0$ for an 
$x : \Obj~(\bullet[a_0,b_0]\cdots[a_n,b_n])$. 

% This is the basic setup for the syntax of weak omega categories,
% obviously more constructors for $\mathsf{Obj}$ are needed which we
% will discussed in the rest of the text. 

\subsection{Composition}\label{sec:composition}
%
Each $n$-cell has $n$ compositions. A 1-cell $f : a \longrightarrow b$
composes with any $g : b \longrightarrow c$ to give $gf : a
\longrightarrow c$. A 2-cell, $\alpha$, composes with $\beta : g
\longrightarrow h : a \longrightarrow b$ to give $\beta\cdot \alpha :
f \longrightarrow h : a \longrightarrow b$ and also with any $\alpha'
: f' \longrightarrow g' : b \longrightarrow c$ to give $\alpha'\alpha
: f'f \longrightarrow g'g : a \longrightarrow c$. Clearly,
compositions are partial functions defined on the subset of
\emph{composable} pairs of cells. The datatype $\meets$ formalises 
composability:
\[
\mathbf{data}\;
\AxiomC{$C~D : \Cat~\Gamma$}
%\AxiomC{$D : \Cat~\Gamma$}
\UnaryInfC{$C \meets D : \Set$}
\DisplayProof
\;\mathbf{where}\;
%\AxiomC{$\{C : \Cat~\Gamma\}$}
%\AxiomC{$\{a~ b~c~ : \Obj~C\}$}
\AxiomC{$\phantom{C}$}
\dblline
\UnaryInfC{$\zeromeets : C[a , b ]  \meets C [ b , c]$}
\DisplayProof~;
\]
\[
% \AxiomC{$\{C~D\}\qquad
% \{a~b : \Obj~C\}$}
% \AxiomC{$\{a'~b' : \Obj~D\}$}
\AxiomC{$H : C ~ \meets ~D $}
\dblline
\UnaryInfC{$\sucmeets{H} : C[a , b] \meets D[a', b']$}
\DisplayProof
\]
%

\begin{remark}[Single-set categories]
  Our approach to $\omega$-categories is essentially
  coinductive. According to it an $\omega$-category, $C$, consists of
  a set of objects, $\Obj~C$, together with an $\omega$-category,
  $C[a,b]$, for each pair of objects, $a$, $b$, of $C$. An alternative
  approach is to say that an $\omega$-category is a set, $X$, of cells
  together with $\omega$-many \emph{category structures}, $C_0, C_1,
  \ldots$ on $X$. What exactly is a \emph{category structure} and
  which axioms it satisfies determines the kind of $\omega$-category
  we are considering.  In a strict $\omega$-category we require a
  category structure to consist of a composition operation and units,
  which satisfy certain equational axioms. In a weak $\omega$-category
  we replace \emph{all} equational axioms with axioms requiring the
  existence of certain \emph{coherence cells}.

  Datatype $\meets$ describes with respect to which category
  structure a composition of cells is taking place. The constructor
  $\zeromeets$ signifies that a composition of cells $f: a
  \longrightarrow b$ and $g: b \longrightarrow c$ is taking place in
  $C_n$. The constructor $\sucmeets{H}$ signifies that the composition
  is taking place in some $C_{n-k}$.
\end{remark}

%
In the following text we elements of $C\,\meets\,D$ just as natural
numbers. Note that it is readily provable by induction, and using
globularity, that 
\mnote{This is essential and we also need the
  opposite direction }
\mnote{Definitions of \textsf{cod} and \textsf{dom} missing}
\begin{align*}
&(C \meets D) \rightarrow (\mathsf{depth}~C \equiv
\mathsf{depth}~D)\\
&(C \meets D) \rightarrow (x : \Obj~C) (y : \Obj~D) \rightarrow \Sigma
(k : \Nat).(\cod~k~x \equiv \dom~k~y)
\end{align*}
%
% 
We can use $\meets$ to define syntax for composition of \emph{cells},
which is mutually recursive with its extension to composition of \emph{categories}: 
%
\[
%\AxiomC{$\{\Gamma\}$}
\AxiomC{$C ~ D : \Cat ~\Gamma$}
\AxiomC{$n : C  \meets D$}
\BinaryInfC{$C \circ_n D : \Cat~\Gamma$}
\DisplayProof
\quad
\AxiomC{$\ditto$}
\AxiomC{$a : \Obj~C\qquad b: \Obj~D$}
\dblline
\BinaryInfC{$a \circ_n b : \Obj~(C \circ_n D)$}
\DisplayProof
\]
Here and later on, $\ditto$ indicates cumulative repetition of all
assumptions from the previous rule.
\[
%\AxiomC{$\{\Gamma\}$}
\AxiomC{$C ~ D : \Cat ~\Gamma$}
\AxiomC{$n : C  \meets D$}
\BinaryInfC{$C \circ_n D : \Cat~\Gamma$}
\DisplayProof
\quad
%\AxiomC{$\{\Gamma\}$}
\AxiomC{$C : \Cat~\Gamma$}
\AxiomC{$a~b~c : \Obj~C$}
\BinaryInfC{$C[a,b] \circ_{0} C[b,c] = C[a,c]$}
\DisplayProof
\]
\[
\AxiomC{$n : C~ \meets~ D$}
\AxiomC{$a~b : \Obj~C$}
\AxiomC{$c~d : \Obj~D$}
\TrinaryInfC{$C[a,b] ~\circ_{n + 1}~ D[c,d] ~=~(C \circ_n D) [ a \circ_n c , b \circ_n d] $}
\DisplayProof
\]
Two examples of composition follow:
\[\bfig
\morphism[a`b;f]
\efig\quad\circ_0\quad
\bfig
\morphism[b`c;g]
\efig
\quad = \quad
\bfig
\morphism[a`c;f \circ_{0} g]
\efig\]
\[\bfig
\morphism(0,1000)/{@{>}@/^2em/}/<1000,0>[a`b;f]
\morphism(0,1000)|b|/{@{>}@/_2em/}/<1000,0>[a`b;g]
\morphism(350,1150)|l|/{@{>}@/_.5em/}/<0,-300>[`;\alpha]
\morphism(650,1150)|r|/{@{>}@/^.5em/}/<0,-300>[`;\alpha']
\morphism(375,1000)|a|<250,0>[`;\gamma]
\morphism/{@{>}@/^2em/}/<1000,0>[a`b;g]
\morphism|b|/{@{>}@/_2em/}/<1000,0>[a`b;h]
\morphism(350,150)|l|/{@{>}@/_.5em/}/<0,-300>[`;\beta]
\morphism(650,150)|r|/{@{>}@/^.5em/}/<0,-300>[`;\beta']
\morphism(375,0)|a|<250,0>[`;\delta]
\place(500,500)[\circ_1]
\efig
\quad
=
\qquad
\bfig
\morphism/{@{>}@/^2em/}/<1000,0>[a`b;f]
\morphism|b|/{@{>}@/_2em/}/<1000,0>[a`b;h]
\morphism(350,150)|l|/{@{>}@/_.5em/}/<0,-300>[`;\alpha\circ_0\beta]
\morphism(650,150)|r|/{@{>}@/^.5em/}/<0,-300>[`;\alpha'\circ_0\beta']
\morphism(375,0)|a|<250,0>[`;\gamma\circ_1\delta]
\efig\]
%
To complement composition, we define identities. 
\[
\AxiomC{$\{C : \Cat~\Gamma\}$}
\AxiomC{$n : \Nat$}
\AxiomC{$b : \Obj~C$}
\TrinaryInfC{$\mathsf{IdCat}~n~b : \Cat~\Gamma$}
\DisplayProof
\quad
\AxiomC{$\ditto$}
\AxiomC{$b : \Obj~C$}
\dblline
\BinaryInfC{$\id\,n\,b : \Obj~(\mathsf{IdCat}\,n\,b)$}
\DisplayProof
\]
\[
\AxiomC{$n : \Nat \qquad b : \Obj~C $}
\UnaryInfC{$\mathsf{IdCat}\,0\,b = \homcat{C}{b}{b}
\quad
\mathsf{IdCat}\,(n+1)\,b = \homcat{(\IdCat\,n\,b)}{\id\,n\,b}{\id\,n\,b}$}
\DisplayProof
\]
\[
\AxiomC{$\phantom{G}\ditto\phantom{G}$}
\UnaryInfC{$\id\,0\,b = b \qquad \id\,(n+1)\,b = \id\,(\id\,n\,b)$}
\DisplayProof
\]

\mnote{worth a formalisation?}
%
For a category $C$ and an object $b$ in $C$ we define simultaneously
iterated identities on $b$, for all $n:\Nat$, and the categories they
lie in, i.e. cells. This is a pattern we see often in the rest of this
paper: \emph{an operator on objects inductively recursively extends to an
  operator on categories}.  We
often write $(\id~n~b)$ as $\id^nb$.

% The definition of $\id$ is omitted for the moment. We could just add
% a new constructor of $\Obj$ but that would collide with other
% coherence cells we introduce later. So we wait and define identities
% as instances of the general principle.

%
% and then express the relationship of the cell on the right-hand side:
% $\gamma\circ_1(\id^1 g)$ to $\gamma$. 
% The notion of relative categories, \emph{telescopes} for short,
% introduced in the next section will help us do that. 



% \begin{ondrej}
%   Note that the composition is not a motivation for telescopes. It is
%   not an obvious motivation because it would be easier to just name
%   $g$ as $\mathsf{cod}^2~\gamma$ and then say that the iterated
%   identity on this meets $C$. Or maybe try to do that and discuss why
%   it ends up being very difficult. 
% \end{ondrej}

\subsection{Relative categories - telescopes}
A strict $\omega$-category is a globular set with compositions and
units that satisfy the unit, associativity and interchange axioms. In
a \emph{weak} $\omega$-category the axioms -- equalities -- are
replaced with data -- \emph{coherence cells}. In order to construct
all required coherence cells we need a finer language for categories
and their cells. The key missing construct is what we call a
\emph{telescope}\footnote{In \cite{}, the word telescope was
  introduced to mean roughly a \emph{dependent context}. We borrow it
  because the nature of our telescopes is similar and, moreover, they
  look like telescopes in pictures.}.

A \emph{telescope} is a category relative to another category
$C$. Formally, we define a new datatype $\Tele$ at the same time as a
\emph{normalisation} function $\Downarrow$:
\[
\AxiomC{$C : \Cat~\Gamma$}
\UnaryInfC{$\Tele~C : \Set$}
\DisplayProof
\quad
\AxiomC{$\{C : \Cat~\Gamma\}$}
\AxiomC{$T : \Tele~C$}
\BinaryInfC{$\cat{T}\, : \Cat~\Gamma$}
\DisplayProof
\]
Telescopes are like categories except that they start from an
arbitrary category $C$ rather than $\bullet$\,: 
\[
\AxiomC{$C : \Cat~\Gamma$}
\dblline
\UnaryInfC{$\telezero{C} : \Tele~C$}
\DisplayProof
\quad
\AxiomC{$T : \Tele~C$}
\AxiomC{$a~b : \Obj~(\cat{T})$}
\dblline
\BinaryInfC{$\telesuc{T}{a}{b} : \Tele ~ C$}
\DisplayProof
\]
Normalisation $\cat{T}$ is defined in the obvious way:
\[
\AxiomC{$T : \Tele~C$}
\AxiomC{$a~b : \Obj~(\cat{T})$}
\BinaryInfC{$\cat{\telezero{C}}~=~C
\qquad
\cat{\telesuc{T}{a}{b}}~=~(\cat{T}) [ a , b ]$}
\DisplayProof
\]
%
We say that an object $x : \Obj~(\cat{T})$ \emph{lies in (the
  telescope) $T$}. When $T$ is relative to a category $C$, $x$ is
called an \emph{object relative to $C$.} 
%
An example of
a telescope is
$\telesuc{\telezero{\homcat{\homcat{\bullet}{a}{b}}{f}{g}}}{\alpha}{\alpha'}$:
\[\bfig
\morphism(350,1150)|l|/{@{>}@/_.5em/}/<0,-300>[`;\alpha]
\morphism(650,1150)|r|/{@{>}@/^.5em/}/<0,-300>[`;\alpha']
\efig
\]
which is relative to
\[\bfig
\morphism(0,1000)/{@{>}@/^2em/}/<1000,0>[a`b;f]
\morphism(0,1000)|b|/{@{>}@/_2em/}/<1000,0>[a`b;g]
\efig
\]
And 
\[\cat{\telesuc{\telezero{\homcat{\homcat{\bullet}{a}{b}}{f}{g}}}{\alpha}{\alpha'}}\quad\equiv\quad\homcat{\homcat{\homcat{\bullet}{a}{b}}{f}{g}}{\alpha}{\alpha'}\enspace.\]
%

It is now straightforward to prove that the category of every
telescope, $T$, lying in a
hom category $\homcat{C}{x}{y}$ meets $\IdCat{(\depth~T)}{x}$ on the left and
$\IdCat{(\depth~T)}{y}$ on the right, where $\depth$ for telescopes is defined in the
obvious way. Formally:
\begin{equation*}%\label{eq:idcat-meets}
%\AxiomC{$\{\Gamma\}\qquad \{C : Cat~\Gamma\}$}
\AxiomC{$a~b : \Obj~C$}
\AxiomC{$T : \Tele~(\homcat{C}{a}{b})$}
\BinaryInfC{$ T~ \mathsf{meets\text{-}id}:(\cat{T}) \meets
  (\IdCat{(\depth~t)}b)\qquad  \mathsf{id\text{-}meets}~T:(\IdCat{(\depth~t)}{a}) \meets (\cat{T})$}
\DisplayProof
\end{equation*}
%
The definition is straightforward and omitted. 
Using these lemmas we can formally define the following
composition:
%
\begin{equation}\label{eq:telescopic-units}
\bfig
\morphism(0,1000)/{@{>}@/^2em/}/<1000,0>[a`b;f]
\morphism(0,1000)|b|/{@{>}@/_2em/}/<1000,0>[a`b;g]
\morphism(350,1150)|l|/{@{>}@/_.5em/}/<0,-300>[`;\alpha]
\morphism(650,1150)|r|/{@{>}@/^.5em/}/<0,-300>[`;\alpha']
\morphism(375,1000)|a|<250,0>[`;\gamma]
\morphism/{@{>}@/^2em/}/<1000,0>[a`b;g]
\morphism|b|/{@{>}@/_2em/}/<1000,0>[a`b;g]
\morphism(350,150)|l|/{@{>}@/_.5em/}/<0,-300>[`;\id g]
\morphism(650,150)|r|/{@{>}@/^.5em/}/<0,-300>[`;\id g]
\morphism(375,0)|a|<250,0>[`;\id^1 g]
\place(500,500)[\circ_1]
\efig
\quad
=
\qquad
\bfig\scalefactor{1.5}
\morphism/{@{>}@/^3em/}/<1000,0>[a`b;f]
\morphism|b|/{@{>}@/_3em/}/<1000,0>[a`b;g]
\morphism(350,150)|m|/{@{>}@/_1em/}/<0,-300>[`;\alpha\circ_0(\id g)\qquad]
\morphism(650,150)|m|/{@{>}@/^1em/}/<0,-300>[`;\qquad\alpha'\circ_0(\id g)]
\morphism(375,0)|a|<250,0>[`;\gamma\circ_1(\id^1 g)]
\efig
\end{equation}


Equality on categories is decidable and so $\meets$ is a
proposition\footnote{Equality on $C\meets D$ satisfies the UIP}. This
justifies our treatment of proofs $p : C \meets D$ as natural numbers
because there is at most one element of $C \meets D$ for each $C$ and
$D$. Of course, we still need to prove, whenever we form a composition
$\alpha \circ_n \beta$, for $\alpha : \Obj~C$, $\beta: \Obj~D$, that
$C \meets D$ holds. But, such proofs are, at least in all cases used in
this text, straightforward by recursion. 
% Figure
% \ref{fig:meetslemmas} summarises all such results needed; we leave it
% for the reader to choose the right one based on the context of each
% composition.

% \begin{figure}[t]
%   \centering
%   $$
%   \AxiomC{$\Gamma : \textsf{Con}\qquad C : \Cat~\Gamma\qquad a ~ b :
%     \Obj ~ C\qquad T : \Tele~(\homcat{C}{a}{b})$}
%   \UnaryInfC{}
%   \DisplayProof
%   $$
%   \begin{align*}
% \mathsf{\mathsf{id\text{-}meets}~T}&~:~(\cat{T}) \meets (\IdCat{(\depth~T)}b)\\
% \mathsf{t~\mathsf{meets\text{-}id}}&~:~(\IdCat{(\depth~T)}{a}) \meets (\cat{T})
%   \end{align*}
%   \caption{Lemmas for $\meets$}
%   \label{fig:meetslemmas}
% \end{figure}

We say that a \emph{telescope $T$ normalises to $C$} when
$((\cat{T})\equiv C)$ is inhabited. Because equality on categories is
decidable, equality of categories is a proposition. Similarly to the
case of $\meets$ we make use of this fact to simplify the syntax and
omit the details of substitutions along proofs $C  \equiv D$. We
introduce the following syntax for substitution along proofs of
category equality:
%
\[
\AxiomC{$\{p : C \equiv C'\}\qquad a : \Obj~C$}
\UnaryInfC{$ \substobj{a} =
  \mathsf{subst}~(\lambda\,X.\Obj~X)~p~a : \Obj~ C'$}
\DisplayProof
\]
%
% Figure \ref{fig:eqlemmas} summarises all equality results required in
% the rest of the text. The one required in any given context is simply
% inferred from types. 
% \begin{figure}[t]
%   \centering
%   $$
%   \AxiomC{$\Gamma : \textsf{Con}\qquad C : \Cat~\Gamma\qquad a ~ b :
%     \Obj ~ C\qquad T : \Tele~(\homcat{C}{a}{b})$}
%   \UnaryInfC{}
%   \DisplayProof
%   $$
%   \begin{align*}
%    \cat{T} &\;\equiv\;\cat{\preptele{a}{b}{T}}
%   \end{align*}
%   \caption{Lemmas for equality on $\Cat$}
%   \label{fig:eqlemmas}
% \end{figure}

%
%
\paragraph{Right-associative telescopes}
%
Telescopes and categories are left-associative iterated
hom-categories, which means it is easy to extend a telescope or
category on the right. To extend telescopes on the left we introduce
the following function: 
%
\[
\AxiomC{$T : \Tele~\homcat{C}{a}{b}$}
\UnaryInfC{$\preptele{a}{b}{T}~:~\Tele~C$}
\DisplayProof
\]
with the definition:
\[
\AxiomC{$T : \Tele~\homcat{C}{a}{b}$}
\UnaryInfC{$\preptele{a}{b}{\telezero{\homcat{C}{a}{b}}}~=~\telesuc{\telezero{C}}{a}{b}\qquad\preptele{a}{b}{(\telesuc{T}{a'}{b'})}~=~\telesuc{(\preptele{a}{b}{T})}{\substobj{a'}}{\substobj{b'}}.$}
\DisplayProof
\]
It holds trivially that:
\[
\AxiomC{$\cat{\preptele{a}{b}{T}} \quad \equiv \quad  \cat{T}$}
\DisplayProof
\]


\section{Generating all coherence cells}
\label{sec:generating}
%
\subsection{Strict equality}
%
A strict $\omega$-category is a globular set with compositions and
units that satisfy the unit, associativity and interchange axioms. In
a \emph{weak} $\omega$-category the axioms -- equalities -- are replaced
with data -- \emph{coherence cells} -- which witness the axioms.

Some examples of coherence cells are the 2-cells:
\begin{align*}
\rho_f &\;:\; \id_a \circ_0 f \Rightarrow f\\
\lambda_f & \; : \; f \circ_0 \id_b \Rightarrow f\\
\alpha_{f,g,h} & \; : \; (f \circ_0 g)\circ_h \Rightarrow f \circ_0 (g
\circ_0 h)
\end{align*}
where the first two witness one half\footnote{such cells are usually
  assumed to be weak equivalences} of the right and left\footnote{note that
  we write compositions the other way round than is usual} identity laws
of composition $\circ_0$ and the last one, $\alpha$ witnesses one-half
of $\circ_0$'s associativity. 
%
To generate all similar laws for all higher compositions, which become
quickly very complicated, we take the following a approach:

We think of the cells of a weak $\omega$-category as living above
cells of a \emph{strict} $\omega$-category via a homomorphism of weak
$\omega$-categories which forgets all non-variable cells and
associations of compositions and sends all coherence cells to
identities. Then for any two cells which in this way
\emph{``strictify''} to the same cell, we insert a coherence
cell. This is in essence the idea of a \emph{contraction} in operadic
definitions of weak $\omega$-categories of Batanin and Leinster\cite{}.
\begin{ondrej}
  A categorical picture would help here: the functor from weak omega
  cats to strict omega cats which is a left adjoint to the inclusion
  of strict into the weak which works, say, by associating everything to the
  left. This makes strict $\omega$-categories reflective in weak $\omega$-cats.
\end{ondrej}

However, to implement this strategy directly would involve defining a
strict $\omega$-category in Type Theory, which would essentially
involve quotienting a weak $\omega$-category.  So we formalise
directly when two cells in an weak $\omega$-category correspond to the
same cell in the underlying strict $\omega$-category.
%
%

To this end we define a proposition $\doteq$ on objects called
\emph{strict equality}. 
%
% The simplest idea is to put:
% \[\AxiomC{$a ~ b : \Obj~C$} 
% \UnaryInfC{$a ~\doteq~ b : \Set$} 
% \DisplayProof\]  
% %
% However, this form proves to be too restrictive.  To
% see why, consider strict equality of the following diagrams in a
% category $C$:
% \begin{equation}\label{eq:lambda}
% \bfig
% \morphism/{@{>}@/^1em/}/[a`b;f] 
% \morphism|m|/{@{>}@/_1em/}/[a`b;f'] 
% \morphism(500,0)/{@{>}@/^1em/}/[b`b;\id_b]
% \morphism(500,0)|m|/{@{>}@/_1em/}/[b`b;\id_b]
% \morphism(0,0)|b|/{@{>}@/_3em/}/<1000,0>[a`b;f']
% \morphism(250,50)|r|/=>/<0,-100>[`;\alpha]
% \morphism(750,50)|r|/=>/<0,-100>[`;\id^2b]
% \morphism(500,-100)|r|/=>/<0,-100>[`;\lambda_{f'}]
% \efig
% \quad\doteq\quad
% \bfig
% \morphism/{@{>}@/^1em/}/[a`b;f] 
% \morphism(500,0)/{@{>}@/^1em/}/[b`b;\id_b]
% \morphism(0,0)|b|/{@{>}@/_3em/}/<1000,0>[a`b;f']
% \morphism(0,0)|a|/{@{>}@/_2em/}/<1000,0>[a`b;f]
% \morphism(500,-200)/=>/<0,-100>[`;\alpha]
% \morphism(750,0)/=>/<0,-100>[`;\lambda_f]
% \efig
% \end{equation}
% %
% As both
% sides correspond to $\alpha$ in a strict $\omega$-category we should
% have that $\mathrm{lhs}\doteq\mathrm{rhs}$. More precisely 
% \[
% \lambda_{f'}\circ_0 (\id^2 b \circ_1 \alpha ) \;\doteq\;\alpha\circ_0\lambda_f
% \]
% However, after a little thought it is obvious that it is not possible
% to break this equality down into a series of simpler equations while
% remaining in the same category, $\homcat{C}{\id_b\circ_0 f }{f'}$. So
% the only option is to include \eqref{eq:lambda} as an axiom.  However
% one obtains a similar and gradually more complicated diagram of
% $n$-cells for each $n\in \Nat$ so such $\doteq$ wouldn't be finitely generated.
% %
%
Any proposition on objects immediately extends to categories and it is
therefore natural to state strict equality of objects with respect to
strict equality of categories they belong to as follows: 
%
\[
\mathbf{data}~
\AxiomC{$C~D : \Cat~\Gamma$}
\UnaryInfC{$C \doteq D : \Set$}
\DisplayProof
\]
\[
\AxiomC{$\mathstrut$}
\UnaryInfC{$\budoteq ~:~ \bullet \doteq \bullet$}
\DisplayProof
\quad
\AxiomC{$ H : C \doteq D $}
\AxiomC{$ A : H \vdash a \doteq c\qquad B :  H \vdash b \doteq d$}
\BinaryInfC{$ \homdoteqcat{H}{A}{B} : \homcat{H}{a}{b} \doteq
  \homcat{H}{c}{d}$}
\DisplayProof
\]
\[
\mathbf{data}~\AxiomC{$H : C \doteq D\qquad
a : \Obj~C\qquad b : \Obj~D $}
\UnaryInfC{$H \vdash a \doteq b : \Set$}
\DisplayProof
~\mathbf{where}~\cdots
\]
%
% We can then simply prove using congruence and only the rule that
% composition of $x$ with $\id$ or $\lambda$ is strictly equal to $x$,
% that :
% \[
%   \lambda_{f'}\circ_0 (\id^2 b \circ_1 \alpha ) \;\doteq\;\id^2 b \circ_1
%   \alpha\;\doteq\;\alpha\;\doteq\;\alpha\circ_0\lambda_f\]
% %
% %
The cases for strict equality on $\Obj$ express the associativity and
left and right unitality of all compositions. Here are the cases for
left units for composition:
%
\[
\AxiomC{$\{T : \Tele~(\homcat{C}{a}{b})\}$}
\AxiomC{$\alpha : \Obj~(\cat{T})$}
\BinaryInfC{$\lambda_{\doteq}~\alpha : \_ \vdash \alpha \circ_{\_}
  (\id^{(\depth~t)}~b) \doteq \alpha$}
\DisplayProof
\quad
\AxiomC{$\{T : \Tele~(\homcat{C}{a}{b})\}$}
\AxiomC{$\alpha : \Obj~(\cat{T})$}
\BinaryInfC{$\rho_{\doteq}~\alpha : \_ \vdash (\id^{(\depth~t)}~a) \circ_{\_}\alpha 
   \doteq \alpha$}
\DisplayProof
\]
\[
\AxiomC{$\{ t : \Tele~(\homcat{C}{a}{b})\}$}
\noLine
\UnaryInfC{$\alpha : \Obj~(\cat{T})$}
\AxiomC{$\{ u : \Tele~(\homcat{C}{b}{c})\}$}
\noLine
\UnaryInfC{$\beta : \Obj~(\cat{u})$}
\AxiomC{$\{ v : \Tele~(\homcat{C}{b}{c})\}$}
\noLine
\UnaryInfC{$\gamma : \Obj~(\cat{v})$}
\TrinaryInfC{$\alpha_{\doteq} : \_ \vdash (\alpha \circ_\_ \beta) \circ_\_ \gamma \doteq \alpha \circ_\_(\beta \circ_\_ \gamma)$}
\DisplayProof
\]

\noindent
Here, underscores replace assumptions about $\meets$ and $\doteq$ on
categories, whose types are inferrable and which are all straightforwardly provable. 


Lastly, we include the following congruence rules:
\[
\AxiomC{$a : \Obj ~C$}
\UnaryInfC{$ \refl{a} : \refl{C} \vdash a \doteq a$}
\DisplayProof
\quad
\AxiomC{$ p : H \vdash a \doteq b $}
\UnaryInfC{$ \sym{p} : \sym{H} \vdash b \doteq a$}
\DisplayProof
\quad
\AxiomC{$p : H \vdash a \doteq b $}
\AxiomC{$q : I \vdash b \doteq c $}
\BinaryInfC{$ \trans{p}{q} : \trans{H}{I} \vdash a \doteq c$}
\DisplayProof
\]
%
\begin{ondrej}
  Make a remark at the beginning how everything defined on objects
  lifts to categories and that we do it without mentioning defining
  everything on objects and extending to categories implicitly.
\end{ondrej}
%
An example is the following strict equality: 
\[
\lambda_{\doteq}~\gamma~:~\_\vdash
\bfig\scalefactor{1.5}
\morphism/{@{>}@/^3em/}/<1000,0>[a`b;f]
\morphism|b|/{@{>}@/_3em/}/<1000,0>[a`b;g]
\morphism(350,150)|m|/{@{>}@/_1em/}/<0,-300>[`;\alpha\circ_0(\id g)\qquad]
\morphism(650,150)|m|/{@{>}@/^1em/}/<0,-300>[`;\qquad\alpha'\circ_0(\id g)]
\efig
\quad \doteq \quad
\bfig
\morphism(0,1000)/{@{>}@/^2em/}/<1000,0>[a`b;f]
\morphism(0,1000)|b|/{@{>}@/_2em/}/<1000,0>[a`b;g]
\morphism(350,1150)|l|/{@{>}@/_.5em/}/<0,-300>[`;\alpha]
\morphism(650,1150)|r|/{@{>}@/^.5em/}/<0,-300>[`;\alpha']
\efig\]
\[\vdash\]
\begin{equation}\label{eq:lambda}
\bfig
\morphism(375,0)|a|<250,0>[`;\gamma\circ_1(\id^1 g)]
\efig\doteq\bfig
\morphism(375,1000)|a|<250,0>[`;\gamma]
\efig
\end{equation}

\paragraph{Strict equality is a proposition}
\begin{ondrej}
\begin{itemize}
\item Is it important whether strict equality is a proposition?
  Perhaps it is a proposition and it is merely a consequence of having
  enough coherence cells (strict equality axioms).
\item What if it wasn't a proposition? Would planes start falling from
  the skies? 
\item Is there a universal characterisation of strict equality?
\end{itemize}
\end{ondrej}
% \begin{ondrej} I believe that without the following we obtain only a
%   ``weak'' version of a weak $\omega$-category,  weak in the sense
%   that there are many cells witnessing the same supposed
%   equality. Though I
%   haven't done the comparison to other definition formally, yet. 
% \end{ondrej}

% \begin{ondrej}
%   $\doteq$ isn't really a proposition. But perhaps we can fix
%   that when we drop reflexivity and symmetry from $\doteq$ and introduce
%   units explicitly. Then $a\doteq b$ is well founded and maybe
%   decidable in the sense that we can construct an element of
%   $a \doteq b$ such that all $h : a \doteq b$ are propositionally equal
%   to $h$, i.e. more succinctly $a \doteq b$ is contractible -- i.e. a
%   proposition. 
% \end{ondrej}
% \[\vdots\]



% For instance the coherence
% cell for the left unit law of $\circ_2$ would have to be a relative
% 4-cell connecting $\gamma$ \eqref{eq:telescopic-units} and the
% composition of $\gamma$ composed before $\id^3$:
% \begin{ondrej} Introduce the term ``relative cells'' together with ``relative
%   categories''
% \end{ondrej}
% \[
% \bfig\scalefactor{1.5}
% \morphism(0,1000)/{@{>}@/^3em/}/<1000,0>[a`b;f\,\circ_0\, \id^0b]
% \morphism(0,1000)|b|/{@{>}@/_3em/}/<1000,0>[a`b;g\,\circ_0\, \id^0b]
% \morphism(350,1150)|l|/{@{>}@/_.5em/}/<0,-300>[`;\alpha\,\circ_1\,\id^2b]
% \morphism(650,1150)|r|/{@{>}@/^.5em/}/<0,-300>[`;\alpha'\,\circ_1\,\id^2b]
% \morphism(375,1000)|a|<250,0>[`;\gamma\,\circ_2\,\id^3b]
% \efig
% \]
% But $\gamma$ and $\gamma\,\circ_2\,\id^3b$ lie in a different category
% (or telescope if we think everything is relative to an ambient category
% $C$).


\subsection{Coherence cells}
\label{sec:witnessing}
%
Having defined when two objects, $a : \Obj~C$, $b : \Obj~D$, are
strictly equal it suffices to add a new \emph{coherence cell},
labelled $\coh~h$, from $a$ to $b$ whenever $a$ and $b$ are strictly
equal via $h : H \vdash a \doteq b$.
%
But wait a minute! The objects $a$ and $b$ are in different categories
so where should $\coh~h$ go?  We must either invent a new object $b'$
in $C$ and define $\coh~h$ to go from $a$ to $b'$, or we must invent a
new object $a'$ in $D$ and define $\coh$ to go from $a'$ go $b$ in
$D$. The third option, which we chose, is to invent a new category, a
pair of objects $a'$, $b'$ in it, and define $\coh$ to go from $a'$ to
$b'$. This is not only the most symmetrical option and corresponding
to categorical definitions of coherence cells in low dimensions, but
it also fits our development the best: Because we understand cells as
pairs $(C,a)$ it makes sense to define coherence cells in the form
$(\homcat{\mathsf{(cohCat~H)}}{a'}{b'} , \coh~h)$ where
$\mathsf{cohCat}~H$ is a new category to which both $a'$ and $b'$ belong.
In the rest of this section we carry out this plan in detail.

% Most rules in the following definitions are cumulative in that all
% assumptions of the rules above are repeated below.  We use ellipsis to
% denote such repeated assumptions. At any rate, the types of all
% unnamed variables can be always inferred from the context.
%
For $H : C \doteq C'$ we define a category $\cohCat~H$:
\[
\AxiomC{$C ~ C' : \Cat~\Gamma$}
\AxiomC{$H : C \doteq C'$}
\BinaryInfC{$ \cohCat~H : \Cat~\Gamma$}
\DisplayProof
\]
Whenever $H : C \doteq C'$, any object $\alpha$ relative to $C$
defines a object relative to $\cohCat~H$. It's denoted
$(\mathsf{subst_1}~H~T~\alpha)$, where $T$ is a telescope in $C$ and
$\alpha : \Obj (\cat {T})$. Likewise, any object $\alpha'$ relative to
$C'$ is defines a object $\mathsf{subst_2}~H~T'~\alpha'$ relative to
$\cohCat~H$.
\[
\AxiomC{$\ditto$}
\AxiomC{$T : \Tele~C$}
\BinaryInfC{$\mathsf{substTele_1}~H~T : \Tele~(\cohCat~H)$}
\DisplayProof
\quad
\AxiomC{$\ditto$}
\AxiomC{$T' : \Tele~C'$}
\BinaryInfC{$\mathsf{substTele_2}~H~T' : \Tele~(\cohCat~H)$}
\DisplayProof
\]
\[
\AxiomC{$\ditto$}
\AxiomC{$\alpha : \Obj ~(\cat{T})$}
\BinaryInfC{$\mathsf{subst_1}~H~T~\alpha :
  \Obj~(\cat{\mathsf{substTele_1}~H~T})$}
\DisplayProof
\quad
\AxiomC{$\ditto$}
\AxiomC{$\alpha' : \Obj ~(\cat{T'})$}
\BinaryInfC{$\mathsf{subst_2}~H~T'~\alpha' :
  \Obj~(\cat{\mathsf{substTele_2}~H~T'})$}
\DisplayProof
\]
%
The
introduced generalisation from mere objects of $C$ and $C'$ to objects
\emph{relative} to $C$ and $C'$ is essential for the recursive
definition of $\cohCat$. Because $\cat{\telezero{C}}= C$ we recover
the special case.

The key ingredient of $\cohCat$, $\mathsf{substTele}_i$ and
$\mathsf{subst}_i$ are coherence cells which we introduce by the
following two new
constructors of $\Obj$ as follows:
%
\[
\AxiomC{$\{C~C'\}$}
\AxiomC{$\{H : C \doteq C'\}\qquad \{a : \Obj~C\}\qquad \{a' :
  \Obj~C'\}$}
\AxiomC{$W : H \vdash a \doteq a'$}
\dblline
\TrinaryInfC{$\coh~W :
  \Obj~(\homcat{(\cohCat~H)}{\mathsf{subst_1}~H~\telezero{C}~a}{\mathsf{subst_2}~H~\telezero{C'}~a'})$}
\noLine
\UnaryInfC{$\coh^{-1}~W :
  \Obj~(\homcat{(\cohCat~H)}{\mathsf{subst_2}~H~\telezero{C'}~a'}{\mathsf{subst_1}~H~\telezero{C}~a})$}
\DisplayProof
\]
%
%
For definitions of $\mathsf{subst}$ and $\mathsf{substTele}$, see
the Appendix. Here we give an unfolding for the example in
\eqref{eq:lambda} to illustrate the recursive pattern and the role of
the key ingredients introduced so far:
%
\[\vdots\]







\section{Interpretation}
\label{sec:interpretation}
\subsection{Globular sets}
A weak $\omega$-category is a globular set, $G$, with certain
constraints on existence of cells. For example, for each object $a$ in
$G$ there must be an object $\id_a$ in $(\mathsf{hom}~G~a~a)$, and for
$x$ in $(\mathsf{hom}~G~a~b)$ and $y$ in $(\mathsf{hom}~G~b~c)$ there
must be an object in $(\mathsf{hom}~G~a~c)$. Abstractly we can
formalise all such constraints by the existence of an action of the
syntax of weak $\omega$-categories on $G$. The details follow:

\begin{definition}
A \emph{weak $\omega$ category} is given by the following data:
\begin{enumerate}
\item A globular set $G : \Glob$ 
\item An action of the syntax of objects on contexts, which are
  lists of cells of $G$:
\[
\AxiomC{$\{\Gamma : \Con\}$}
\AxiomC{$\{C : \Cat~\Gamma\}$}
\AxiomC{$o : \Obj~C\qquad x : \intpr{\Gamma}$}
\TrinaryInfC{$\intpr{o}~x : \mathsf{obj}~(\intpr{C}~x)$}
\DisplayProof
\]
where the following extensions of $\intpr{\text{-}}$ to contexts,
categories and variables are used:
\[
\AxiomC{$\Gamma : \Con$}
\UnaryInfC{$\intpr{\Gamma} : \Set$}
\DisplayProof
\quad
\AxiomC{$\phantom{\Gamma}$}
\UnaryInfC{$\intpr{\varepsilon} = 1$}
\DisplayProof
\quad
\AxiomC{$\phantom{\Gamma : \Con}$}
%\AxiomC{$C : \Cat~\Gamma$}
\UnaryInfC{$\intpr{\Gamma , C} = \Sigma(x : \intpr{\Gamma})(\intpr{C} ~ x)$}
\DisplayProof
\]
\[
\AxiomC{$C : \Cat ~ \Gamma$}
\AxiomC{$x : \intpr{\Gamma}$}
\BinaryInfC{$\intpr{C}~x : \Glob$}
\DisplayProof
\quad
\AxiomC{$\phantom{\intpr{\Gamma}}$}
\UnaryInfC{$\intpr{\bullet}~x = G$}
\DisplayProof
\AxiomC{$\phantom{\intpr{\Gamma}}$}
\quad
\UnaryInfC{$\intpr{\homcat{C}{a}{b}}~x = \mathsf{hom}~(\intpr{C}
  x)~(\intpr{a} ~ x)~(\intpr{b}~x)$}
\DisplayProof
\]

\[\vdots\]\mnote{to do: variables as projections}
and such that
\[
\AxiomC{$\{C : \Cat~\Gamma\}$}
\AxiomC{$v : \mathsf{Var}~C$}
\BinaryInfC{$\intpr{\var~v} ~=~ \intpr{v}~\intpr{\Gamma}$}
\DisplayProof
\qquad
\AxiomC{$\phantom{x : \Obj~C}$}
\AxiomC{$\phantom{D : \Cat~\Gamma}$}
\BinaryInfC{$\intpr{\wk~x~D}  ~=~ \intpr{x}$}
\DisplayProof
\]
\end{enumerate}
\end{definition}
%
%
\subsection{Initiality}
\begin{ondrej}
  I feel this is important, but must yet fill in the details. Maybe
  because in the operadic approaches, one is looking for an
  algebra of a the initial contractible globular operad. 

  At any rate, we want to have the free weak $\omega$-category on a
  given context and this should be initial.
\end{ondrej}
%
We show that equality on $\Sigma(C : \Cat~\Gamma) (\Obj~C)$ is
contractible and that the identity morphism is an interpretation.  But
we already know that it's the case because equality on $\Obj~C$ is
decidable.
%
\mnote{More detail is needed.}
%
% This shows the globular set $\Sigma(C : \Cat~\Gamma)(\Obj~C)$ (see
% \ref{}) is the initial weak $\omega$-category....
% \begin{ondrej}
%   \textsc{Total rubbish}! what is the real connection with initiality?
% \end{ondrej}

\begin{ondrej}
  Deepen and reference the connection with the syntactical development
  of globular sets and meets.
\end{ondrej}

\subsection{Example: the weak $\omega$-groupoid of a type}



\end{document}
